\chapter{Fundamentação Teórica}
\label{cap:fund}

\section{Definições Principais}

Para melhor esclarecer os assuntos abordados, é importante que seja primeiramente definido alguns dos termos centrais para a fundamentação do trabalho.

\subsection{Falhas}

Um erro (\textit{error}) é a diferença entre um valor esperado e o valor obtido. Um defeito (\textit{fault}) é um estado irregular do sistema que pode (ou não) provocar erros que resultem em falhas. Já uma falhas (\textit{failure}) é uma incapacidade observável do sistema de cumprir sua função designada, constituindo uma degradação total ou parcial de sua qualidade de serviço @IEEE.

Durante este trabalho, o termo "falha" será utilizado de forma mais geral, representando um estado ou evento no sistema que cause uma degradação na sua qualidade de serviço.

\subsection{Qualidade de Serviço}

Qualidade de serviço é uma métrica sistêmica que sumariza o quão bem o sistema provisiona suas funcionalidades em um determinado momento. É possível definir e modelar essa métrica de diversas maneiras, mas para os própositos deste trabalho, será utilizada uma definição simples que resume a funcionalidade geral numa escala de 0 a 1.

A qualidade de serviço $Q$ do sistema pode ser aproximada pela média ponderada de seus serviços $S_0 ... S_n$ com os pesos de seus fatores de contribuição para a qualidade total $q_0 ... q_n$ @SchedAndOptOfDistributedFT.

\begin{equation}
    Q = \frac{ \sum^{i = 0}_{n} S_i q_i }{ \sum^{i = 0}_{n} q_i}
\end{equation}

\subsection{Dependabilidade}

Será utilizado o termo dependabilidade como uma propriedade que sumariza os atributos:  confiabilidade, disponibilidade, capacidade de manutenção e segurança (conhecidos em inglês como critérios "RAMS". Os critérios serão definidos na seção seguinte.

A tolerância à falhas impacta positivamente os critérios confiabilidade e disponibilidade, e pode em alguns casos melhorar a capacidade de manutenção, sendo assim, a tolerância à falhas é um aspecto importante para sistemas com boa dependabilidade.

\subsubsection{Confiabilidade}

Confiabilidade (\textit{Reliability}), é a probabilidade de um sistema executar corretamente no período $[t_0, t]$. Para modelar essa métrica é necessário um modelo estatístico que é particular da aplicação. A confiabilidade $R$ é uma função do tempo $t$, a taxa de falhas $\lambda$ e quaisquer sejam os outros parâmetros do modelo. @FaultInjectionTechniques

\begin{equation}
    R(t) = f(t, \lambda, ...)
\end{equation}

\subsubsection{Disponibilidade}

Disponibilidade (\textit{Availability}) é a razão entre o tempo em que o sistema não consegue prover seu serviço (\textit{downtime}) e o e seu tempo total de operação @FaultInjectionTechniques. A disponibilidade $A$ pode ser modelada em termos do tempo disponível $t_{up}$ e do tempo indisponível $t_{down}$:

\begin{equation}
    A = t_u / (t_{up} + t_{down})
\end{equation}

% \renewcommand{\arraystretch}{1}
%  \begin{quadro}[H]
%     \centering
%     \caption{Comparação de trabalhos relacionados}
%     \label{tab:trabrel}
%     \begin{tabular}{|m{0.126\textwidth}|m{0.135\textwidth}|m{0.10\textwidth}|m{0.115\textwidth}|m{0.11\textwidth}|m{0.18\textwidth}|}
%         \hline
%         \rowcolor[HTML]{C0C0C0}
%         \textbf{Trabalho}  & \textbf{CNN base} & \textbf{Imagem} & \textbf{Tipo de Resolução} & \textbf{$\mu$C} & \textbf{Métricas} \\ \hline
        
%         \Centering\textbf{\citeDir{azami2022earth}}  & ShallowNet, LeNet e MiniVGGNet & RGB & Espacial e Espectral & Raspberry Pi 3+ & Acurácia \\ \hline
        
%         \Centering\textbf{\citeDir{maskey2020cubesatnet}}  & CubeSatNet & RGB & Espacial e Espectral & STM32H & Acurácia Global, \textit{F1-score} e Memória \\ \hline 
        
%         \Centering\textbf{\citeDir{leong2021unet}}  & U-net & RGB & Espacial e Espectral & STM32F7 & Precisão, Sensibilidade, Falso Negativo, Falso Positivo, Acurácia Global, \textit{F1-score} e Memória \\ \hline
        
%         \Centering\textbf{Este trabalho}  & LeNet-5 e HybridSN & HSI & Espectral & ESP32 e/ou Raspberry Pi Pico & Acurácia, F1-Score, Sensibilidade, Tempo de Processamento, Memória Utilizada, Potência Dissipada e Energia Consumida\\ \hline
%     \end{tabular}
%  \end{quadro}
%  \renewcommand{\arraystretch}{1}
 
