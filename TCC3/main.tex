\documentclass[
    % -- opções da classe memoir --
    12pt,                 % tamanho da fonte
    openright,            % capítulos começam em pág ímpar (insere página vazia caso preciso)
    oneside,              % para impressão em um lado. Oposto a twoside, verso e anverso
    a4paper,              % tamanho do papel. 
    % -- opções da classe abntex2 --
    chapter=TITLE,        % títulos de capítulos convertidos em letras maiúsculas
    section=TITLE,        % títulos de seções convertidos em letras maiúsculas
    % -- opções do pacote babel --
    english,              % idioma adicional para hifenização
    brazilian,            % o último idioma é o principal do documento
]{styles/abntex2}


\usepackage[alf,
    %versalete,
    abnt-emphasize = bf, % destaca o titulo da revista ou livro em negrito;
    abnt-etal-list = 3, % trabalhos com mais de 3 autores recebem et al.,;
    %abnt-etal-text = it, % escreve o et al., em italico;
    bibliographystyle = styles/abntex2-alf.bst,
    abnt-repeated-author-omit = no % autores com + de uma entrada recebem '____.'
]{styles/abntex2cite}

\usepackage[T1]{fontenc}
\usepackage[utf8]{inputenc}

%\usepackage[dvipsnames]{xcolor}
\usepackage[table,xcdraw,dvipsnames]{xcolor}
%\usepackage{subfigure}   % Pacote para subfiguras
\usepackage{amsmath}
\usepackage{lastpage}
\usepackage{times}
\usepackage{graphicx}
\usepackage{caption}
\usepackage{tocloft}
\usepackage{tabularx}
\usepackage{booktabs}
\usepackage{enumitem}
\usepackage{dirtytalk}
\usepackage{tikz}
\usetikzlibrary{tikzmark}
%\usepackage{array} 
\usepackage{multirow}
\usepackage{threeparttable} % to use table notes

\usepackage{caption}
\usepackage{textcomp}


%%%%%%%%%%%%%%%%%%%%%%%%%%%%%%%%%%%%%%%%%%%%%%%%%%%%%%%%%%%%%%%%%
%%%%%%%%%%%%%%%%%%%%%%%%%%%%%%%%%%%%%%%%%%%%%%%%%%%%%%%%%%%%%%%%%
%%%%%%%%%%%%%%%%%%%%%%%%%%%%%%%%%%%%%%%%%%%%%%%%%%%%%%%%%%%%%%%%%
\usepackage{listings}
\usepackage{xcolor}

%New colors defined below
\definecolor{codegreen}{rgb}{0,0.6,0}
\definecolor{codegray}{rgb}{0.5,0.5,0.5}
\definecolor{codepurple}{rgb}{0.58,0,0.82}
\definecolor{backcolour}{rgb}{0.95,0.95,0.92}

% Code listing style named "mystyle"
\lstdefinestyle{mystyle}{
    backgroundcolor=\color{backcolour}, commentstyle=\color{codegreen},
    keywordstyle=\color{magenta},
    numberstyle=\tiny\color{codegray},
    stringstyle=\color{codepurple},
    basicstyle=\ttfamily\footnotesize,
    breakatwhitespace=false,         
    breaklines=true,                 
    captionpos=b,                    
    keepspaces=true,                 
    numbers=left,                    
    numbersep=5pt,                  
    showspaces=false,                
    showstringspaces=false,
    showtabs=false,                  
    tabsize=2
}

%"mystyle" code listing set
\lstset{style=mystyle}
%%%%%%%%%%%%%%%%%%%%%%%%%%%%%%%%%%%%%%%%%%%%%%%%%%%%%%%%%%%%%%%%%

% Criação do symbol da função sign
\DeclareMathOperator{\sign}{sign}

\newcolumntype{C}[1]{>{\centering\let\newline\\\arraybackslash\hspace{0pt}}m{#1}}
\newcolumntype{L}[1]{>{\RaggedRight\let\newline\\\arraybackslash\hspace{0pt}}m{#1}}

% Alinhamento de legenda à esquerda
\captionsetup{justification=raggedright,singlelinecheck=false}

\usepackage{styles/ttc_univali}

\begin{document}
\newcounter{equationCounter}

\pretextual

% Capa e elementos pre-textuais
\begin{Info}
% Universidade
{UNIVERSIDADE DO VALE DO ITAJAÍ}
% Escola
{ESCOLA POLITÉCNICA}
% Curso
{CURSO DE CIÊNCIA DA COMPUTAÇÃO}
% Titulo
{DETECÇÃO DE ERROS EM SISTEMA OPERACIONAL DE TEMPO REAL}
% Autor
{Marcos Augusto Fehlauer Pereira}
% Cidade e Data
{Itajaí (SC), Junho de 2025}
% Nome da Área de concentração
{Sistemas Operacionais}
% Orientador(a)
{Felipe Viel, MSc.}
% Coorientador(a) <Nome do Coorientador(a)>, <Titulação> %%%%%%%% Se não tiver coorientador deixe vazio

\end{Info}

%\begin{Dedicatoria}
% Dedicatória
% \end{Dedicatoria}

% \begin{Agradecimentos}
% Agradeço a todos.
% \end{Agradecimentos}

% \begin{Epigrafe}
% Epígrafe
% \end{Epigrafe}

\begin{Resumo}
Sistemas embarcados são sistemas especializados tipicamente encontrados como um componente lógico de um dispositivo maior, estes sistemas utilizam-se com frequência um tipo especializado de sistema operacional: Sistemas operacionais de tempo real, que permitem que múltiplas tarefas executem de forma concorrente. Em diversas situações é necessário que esses dispositivos operem em condições adversas (como radiação ionizante e interferência eletromagnética) que alteram seu comportamento esperado e degradam sua qualidade de serviço, para operar dentro de tais condições, técnicas de tolerância à falhas são aplicadas, que visam permitir a operação razoável do sistema mesmo na presença de falhas. Para viabilizar tolerância à falhas é possível utilizar de diversos mecanismos, dentre eles, aqueles que operam em conjunto com o escalonador do sistema operacional de tempo real podem permitir um grau de resiliência maior enquanto visando reduzir a ociosidade dos núcleos da máquina. Este trabalho visa explorar e aplicar técnicas de tolerância e detecção de falhas (Redundância Modular, Reexecução, Heartbeat Signal, CRCs e Asserts) com uma interface voltada ao escalonador do sistema operacional, com o objetivo de fornecer uma análise dos custos e vantagens associados à cada combinação de técnicas.

\textbf{Palavras-Chave}: Sistemas Embarcados. Sistemas Operacionais. Tolerância a Falhas. FreeRTOS. Escalonador.

\end{Resumo}

\begin{Abstract}

\textit{Embedded Systems are specialized systems that are typically found as a logical component of a greater device, these systems are frequently equipped with a special kind of operating system: a Real Time Operating System, that allow for multiple tasks to be executed concurrently. In many situations, it is required that such devices operate in adverse or volatile conditions (e.g. ionizing radiation and electromagnetic interference) that alter its behavior, causing a degradation in its quality of service. Thus, to be able to operate within these contexts, fault tolerance techniques are applied, with the goal of allowing reasonable system operation even within the presence of faults. Many mechanisms may be used to achieve fault tolerance, among them, there are those that operate in conjunction with the real time operating system's scheduler to offer a greater degree of reliability while also decreasing idle time of them machine's cores. This work shall explore and apply techniques of fault tolerance (Modular Redundancy, Re-execution, Heartbeat Signal and Asserts) that have an interface focused on the scheduler's capabilities, with the main objective of providing an analysis over the tradeoffs attached to permutations of those techniques.}

\textit{\textbf{Keywords}: Embedded Systems. Operating Systems. Fault Tolerance. FreeRTOS. Scheduler.}

\end{Abstract}

 \clearpage
% ---
% inserir lista de ilustrações
% ---
\pdfbookmark[0]{\listfigurename}{lof}
\listoffigures*
\cleardoublepage
% ---

% ---
% inserir lista de tabelas
% ---
% \pdfbookmark[0]{\listtablename}{lot}
% \listoftables*
% \cleardoublepage
% ---

% ---
% inserir lista de quadros
% ---
\pdfbookmark[0]{\listofquadrosname}{loq}
\listofquadros*
\cleardoublepage
% ---
\pdfbookmark[0]{\listofequacaoname}{loe}
\listofequacao*
\cleardoublepage

% ---
% inserir lista de abreviaturas e siglas
% ---
\begin{siglas}
    \item[COTS]  Commercial Off-The-Shelf
    \item[CRC]   Cyclic Redundancy Check
    \item[CPU]   Central Processing Unit
    \item[FT]    Fault Tolerance
    \item[FFT]   Fast Fourier Transform
    \item[GDB]   GNU Debugger
    \item[iFFT]  Inverse Fast Fourier Transform
    \item[IoT]   Internet of Things
    \item[ITAR]  International Traffic in Arms Re-gulations
    \item[iSCSI] Internet Small Computer Systems Interface
    \item[PCB]   Printed Circuit Board
    \item[QEMU]  Quick EMUlator
    \item[RTOS]  Real Time Operating System
    \item[RAMS]  Reliability, Availability, Mantainability, Safety
    \item[SCTP]  Stream Control Transmission Protocol
    \item[TCC]   Trabalho de Conclusão de Curso
\end{siglas}
% ---

% ---
% inserir o sumario
% ---
\pdfbookmark[0]{\contentsname}{toc}
\tableofcontents*
\cleardoublepage
% ---

 \clearpage

\textual % Indica inicio dos elementos textuais
\pagestyle{simple} % remove o cabecalho

\chapter{Introdução}
\label{cap:intro}


A tolerância a falhas (TF) refere-se à capacidade de um sistema computacional de manter sua qualidade de serviço mesmo na presença de estados adversos. Esta propriedade tornou-se cada vez mais crítica com a expansão da computação para ambientes hostis e aplicações de missão crítica. Sistemas distribuídos, por exemplo, são particularmente vulneráveis a falhas em seus canais de comunicação, que estão sujeitos a degradação ou inoperabilidade total devido à interferência eletromagnética, falta de energia e eventos climáticos \cite{FaultTolerantSystems}.

Sistemas embarcados, que constituem o foco deste trabalho, podem enfrentar desafios mais complexos, além das falhas de comunicação supracitadas, estes sistemas podem ter seus dados em memória ou registradores diretamente corrompidos por causas externas como: radiação ionizante, flutuações extremas de temperatura ou voltagem, e colisões físicas  \cite{DependabilityInEmbeddedSystems}.

O contexto atual da indústria adiciona camadas adicionais de complexidade ao problema. Nos últimos anos, observou-se uma maior adoção de sistemas COTS (Commercial Off-The-Shelf) em aplicações que tradicionalmente utilizariam hardware especializado \cite{CyberSecSpaceCOTS}. Esta tendência é motivada tanto por fatores econômicos quanto por restrições regulatórias, como as impostas pelo ITAR (International Traffic in Arms Regulations) dos Estados Unidos, que limitam a exportação de certas tecnologias \cite{ITARPCBCompliance}. Consequentemente, fabricantes e desenvolvedores em diversos países precisam implementar tolerância à falhas através de técnicas em software, já que soluções robustas em hardware nem sempre estão disponíveis em sistemas COTS.

A presença dos fenômenos adversos mencionados exige que os dispositivos estejam adequadamente preparados, especialmente em aplicações críticas onde as consequências de um erro podem ser catastróficas. Exemplos incluem sistemas de controle automotivo, equipamentos médicos, sistemas de navegação aeroespaciale infraestrutura energética e de telecomunicações  \cite{FaultTolerantSystems}.

Tornar um sistema tolerante à falhas é um problema multi facetado, ambas soluções em hardware e software necessitam ser abordadas para garantir a qualidade de serviço desejada. O escalonador é o componente crucial de um sistema operacional para a execução concorrente de diversas tarefas \cite{OperatingSystemConcepts}. O processo de detecção das falhas e seu custo de tempo de CPU e uso de memória deve ser considerado pois a reação rápida e correta às falhas requer previamente a detecção e elaboração das rotinas de escalonamento de forma adequada \cite{DependabilityInEmbeddedSystems}.

Este trabalho visa portanto fazer uma análise do impacto de diferentes técnicas de tolerância à falhas durante o escalonamento de tarefas e seu impacto na performance de um sistema, para que se possa melhor compreender e evidenciar os custos e benefícios ao tornar um sistema mais resiliente. 

\section{Problematização}

Sistemas embarcados estão presentes em diversos contextos como: exploração espacial, indústria automotiva, tecnologia médica, dispositivos de baixo consumo energético e sistemas distribuídos no contexto IoT. Portanto, entende-se que manter alto grau da qualidade de serviço com o mínimo de degradação de performance e aumento de custo (monetário ou energético), pode prover uma vantagem econômica para fabricantes e provedores assim como um benefício social na maior confiabilidade no caso de aplicações críticas. \cite{DependabilityInEmbeddedSystems}.

Ademais, ocorreu nos últimos anos uma maior adoção de sistemas COTS (Commercial off the shelf), por poderem ser economicamente mais viáveis e fornecerem uma solução "genérica" para problemas que anteriormente necessitariam de hardware especializado \cite{CyberSecSpaceCOTS}. E mesmo caso pretenda-se utilizar um design especializado para o produto, estes sistemas são úteis para a fase de prototipação e validação do projeto, dado sua facilidade de acesso e flexibilidade.

Um outro fator que influencia na adoção do uso de COTS para certas aplicações que necessitam de tolerância à falhas são as regulações ITAR (International Traffic in Arms Regulations) imposta pelos Estados Unidos que restringe a exportação de diversas tecnologias de cunho potencialmente militar. Neste caso, o impedimento da exportação certos tipos  tecnologias de PCBs e firmware aumenta mais ainda a necessidade de compradores de outros países adquirirem alternativas comerciais mais comuns que já são complacentes com a regulação \cite{ITARPCBCompliance}.

O custo de utilizar técnicas de tolerância é sensível ao contexto da aplicação e ao nível de tolerância desejado, e no caso dos sistemas COTS, técnicas robustas de resiliência em hardware nem sempre estão disponíveis, sendo necessário delegar esta funcionalidade para a aplicação. É necessário escolher a técnica de tolerância mais adequada de uma multidão de técnicas. Para que a escolha seja informada, é essencial que as trocas em termos de performance e grau de confiabilidade adquiridos sejam compreendidos para minimizar o custo de manter uma qualidade de serviço desejada.

\subsection{Solução Proposta}

Implementar e comparar técnicas de escalonamento com detecção de erros, com o objetivo de esclarecer o impacto de performance em relação ao ganho de dependabilidade do sistema, particularmente no contexto de sistemas com restrição \textit{Hard Real Time}, pois se uma técnica é capaz de satisfazer o critério de tempo real mais rígido, também poderá ser usada em contextos com critério temporal mais relaxado.

\section{Objetivos}

Esta seção formaliza os objetivos do trabalho, conforme descrito a seguir.

\bigskip
\subsection{Objetivo Geral}

Analisar o impacto de técnicas de tolerância à falhas em software num sistema operacional de tempo real.

\bigskip
\subsection{Objetivos Específicos}

\begin{enumerate}
    \item Selecionar técnicas de tolerância à falhas em software
    \item Implementar técnicas escolhidas com uma interface para uso
    \item Realizar testes com de injeção de falhas e coletar métricas de performance das técnicas
    \item Produzir uma análise comparativa das técnicas, seus custos e eficácia
\end{enumerate}

\section{Metodologia}

O objetivo do trabalho é descritivo e exploratório, as métricas coletadas são de caráter quantitativo e conclusões e observações derivadas do trabalho serão realizadas de maneira indutiva baseadas nas métricas de performance coletadas e comparadas.

Foi realizado uma pesquisa bibliográfica para a fundamentação e escolha das técnicas e dos materiais do trabalho, sendo esta primariamente focada em autores com obras associadas ao tema de tolerância assim como temas adjacentes relevantes como sistemas operacionais e interface hardware-software.

Após isso será realizado uma implementação e testes das técnicas escolhidas para validação, e uma campanha de injeção de falhas será realizada em um microcontrolador para a coleta final das métricas no \autoref{cap:proj}.

\section{Estrutura do Trabalho}

No \autoref{cap:fund} serão explorados os tópicos centrais que constituem a premissa do trabalho, primariamente conceitos de detecção de falhas e escalonamento nos sistemas operacionais. Sistemas operacionais são um tópico particularmente vasto, portanto sua abordagem será focada apenas nos aspectos mais essenciais para o trabalho. No \autoref{cap:proj} é definido o projeto e o plano de verificação para a implementação e análise das técnicas descritas seguido de sua implementação no \autoref{cap:desenvolvimento}. No \autoref{cap:resultados} são expostos os resultados dos testes realizados e no \autoref{cap:conclusoes} é realizado uma recapitulação do que foi abordado no trabalho e quais temas podem ser expandidos por trabalhos futuros

 \clearpage
\chapter{Fundamentação Teórica}
\label{cap:fund}

Este capítulo apresenta os conceitos fundamentais necessários para a compreensão do trabalho. Inicialmente são definidos termos essenciais. Em seguida, são exploradas as técnicas de tolerância à faltas em nível conceitual. Por fim, são abordados aspectos de sistemas embarcados e sistemas operacionais de tempo real e quais critérios se desejam dos mesmos.

\section{Definições Principais}

Para melhor esclarecer os assuntos abordados, é importante que seja primeiramente definido alguns dos termos centrais para a fundamentação do trabalho.

\subsection{Faltas}

De acordo com a definição de anormalidades de software da IEEE: um erro (\textit{error}) é a diferença entre um valor esperado e o valor obtido. Um defeito (\textit{fault}) é um estado irregular do sistema que pode (ou não) provocar erros que resultem em faltas. Já uma falha (\textit{failure}) é uma incapacidade observável do sistema de cumprir sua função designada, constituindo uma degradação total ou parcial de sua qualidade de serviço \cite{IEEEAnormalities}.

Neste trabalho, o termo "falta" será utilizado de forma mais geral como uma tradução alternativa da palavra inglesa "*fault*", representando um estado ou evento no sistema que cause uma degradação na sua qualidade de serviço.

\subsection{Padrões de Ocorrência}

Faltas podem ser classificadas em 3 grupos principais quanto ao seu padrão de ocorrência \cite{FaultTolerantSystems}.

\begin{itemize}
    \item Faltas Transientes: Ocorrem aleatoriamente e possuem um impacto temporário.

    \item Faltas Intermitentes: Assim como as transientes possuem duração e impacto temporários, porém ocorrem periodicamente.

    \item Faltas Permanentes: Causam uma degradação permanente no sistema da qual não pode ser recuperada, potencialmente necessitando de intervenção externa.
\end{itemize}

\subsection{Dependabilidade}

Será utilizado o termo dependabilidade como uma propriedade que sumariza os atributos:  confiabilidade, disponibilidade, capacidade de manutenção e segurança (conhecidos em inglês como critérios RAMS). Os critérios serão definidos na seção seguinte.

A tolerância à faltas impacta positivamente os critérios confiabilidade e disponibilidade, e pode em alguns casos melhorar a capacidade de manutenção, portanto a tolerância à faltas é um aspecto importante para sistemas com dependabilidade.

\subsection{Confiabilidade}

Confiabilidade (\textit{Reliability}), é a probabilidade de um sistema executar corretamente no período $[t_0, t]$. Para modelar essa métrica é necessário um modelo estatístico que é particular da aplicação. A confiabilidade $R$ é uma função do tempo $t$, a taxa de faltas $\lambda$ e quaisquer sejam os outros parâmetros do modelo \cite{FaultInjectionTechniques}.

\begin{equation}
    R(t) = f(t, \lambda, ...)
\end{equation}
\addEquacao{Confiabilidade}{2}

\subsection{Disponibilidade}

Disponibilidade (\textit{Availability}) é a razão entre o tempo em que o sistema não consegue prover seu serviço (\textit{downtime}) e o e seu tempo total de operação \cite{FaultInjectionTechniques}. A disponibilidade $A$ pode ser modelada em termos do tempo disponível $t_{up}$ e do tempo indisponível $t_{down}$:

\begin{equation}
    A = t_{up} / (t_{up} + t_{down})
\end{equation}
\addEquacao{Disponibilidade}{3}

\subsection{Capacidade de manutenção}

Capacidade de manutenção (\textit{Maintainability}) é a probabilidade de um sistema em um estado inválido ser reparado com sucesso antes de um tempo $t$ \cite{FaultInjectionTechniques}.

A modelagem deste atributo necessita de conhecimento particular sobre a aplicação e sobre a disponibilidade de equipamentos ou especialistas humanos para a realização do reparo. Pode ser definida como uma função probabilidade do tempo $t$, taxa de faltas $\lambda$ e os outros parâmetros do modelo.

\begin{equation}
    M(t) = f(t, \lambda, ...)
\end{equation}
\addEquacao{Capacidade de Manutenção}{4}

\subsection{Segurança}

Segurança (\textit{Safety}) é a probabilidade do sistema não causar danos à integridade humana ou à outros patrimônios, independentemente da presença faltas. Por ser um critério muito particular da natureza da aplicação e seu contexto de operação, uma estimativa analítica necessita de um modelo estatístico que não é facilmente sumarizado com apenas uma equação \cite{FaultInjectionTechniques}.

\section{Tolerância à Faltas}

\subsection{Mecanismos de Detecção}

Mecanismos de detecção são responsáveis por identificar a presença de uma falta no sistema, um bom mecanismo de detecção deve ser capaz de detectar uma classe grande de faltas sem introduzir uma penalidade grande ao tempo de execução do sistema. Dentre as possíveis escolhas de mecanismos, os subsequentes são utilizados neste trabalho.

\subsection{CRC (Cyclic Redundancy Check)}

Os CRCs são códigos de detecção de erro comumente utilizados em redes de computador e armazenamento não volátil. Para cada segmento de dado é concatenado um valor de checagem que é calculado com base no resto da divisão com um polinômio gerador pré definido \cite{FaultTolerantSystems}.

CRCs são comumente utilizados devido à serem simples de implementar, ocuparem pouco espaço adicional no segmento de dados e serem resilientes à erros em rápida sucessão, faltas transientes que alteram uma região de bits próximos.

\subsection{Heartbeat signals}

Os \textit{heartbeat signals} (sinais de batimento cardíaco) são sinais periódicos para garantir a atividade de um nó computacional com o recebimento de uma resposta, são também chamados de \textit{watchdog timers} \cite{DependabilityInEmbeddedSystems}. Uma das aplicações desta técnica em conjunto com reexecução ou replicação é a possibilidade de detectar uma demora excessiva e preventivamente cancelar uma das unidades de execução.

\begin{figure}[H]
    \centering
    \captionsetup{justification=centering}
    \caption{Sequência de um Heartbeat Signal}
	\figuraborda{width=1.0\textwidth}{assets/heartbeat_signal.png}\\
    \captionsetup{justification=raggedright}
    \caption*{Fonte: Elaborada pelo autor}
    \label{fig:heartbeatSignal}
\end{figure}

Na \autoref{fig:heartbeatSignal}, é utilizado a resposta tardia para deduzir a presença de uma falta na tarefa ou no canal de transmissão. Estes sinais também podem ser utilizados no contexto de tempo real para a validação de um prazo ou sub prazo da tarefa \cite{FaultTolerantSystems}.

\subsection{Asserts}

Asserts são mecanismos simples e flexíveis para a detecção de faltas, consistindo na verificação de uma condição que, durante uma execução normal do programa, deve permanecer invariavelmente verdadeira (denominada "invariante"). Caso a invariante seja falsa, detecta-se a presença de uma falta. A \autoref{fig:assert} ilustra o fluxo de um assert.

\begin{figure}[H]
    \centering
    \captionsetup{justification=centering}
    \caption{Fluxograma de um Assert}
	\figuraborda{width=0.5\textwidth}{assets/assert\_diagram.png}
    \captionsetup{justification=raggedright}
    \caption*{Fonte: Elaborada pelo autor}
    \label{fig:assert}
\end{figure}

Apesar de sua simplicidade, quando usados em conjunto com simulações determinísticas e ferramentas de \textit{fuzzying}, asserts podem detectar erros durante a execução assim como revelar erros de design durante a fase de desenvolvimento \cite{TigerBeetleSafety} \cite{PowerOf10Rules}.

Durante a execução de um sistema tolerante à faltas, asserts servem como uma forma de saber rapidamente que algo errado aconteceu. Porém não são robustos o suficiente para detectar corrupção silenciosa de dados ou pulos inesperados de maneira consistente. Quando asserts são inseridos na entrada ou saída de um procedimento são denominados como pré-condições e pós-condições respectivamente. Alguns compiladores são capazes de automaticamente inserir estas condições para assegurar contratos da interface de um programa. Algumas linguagens de programação, utilizam de uma forma automática de asserts chamados de "contratos", que podem ser usados para garantir certas pós e pré condições, certos compiladores como o da linguagem SPARK fazem uso destas capacidades para realizar verificação formal de um programa \cite{SPARKContracts}.

\subsection{Mecanismos de Tratamento}

Uma vez que uma falta tenha sido detectada o sistema precisa tratar a falta o mais rápido possível para manter a qualidade de serviço, alguns mecanismos de detecção também fornecem a possibilidade de correção dos dados, como é o caso dos códigos Reed-Solomon, nestes casos, fica à critério da aplicação se a correção deve ser tentada ou outro tratamento deve ser usado.

\subsection{Re-execução}

Re-executar uma tarefa é uma outra forma simples de recuperar-se de uma falta, a probabilidade de $k$ faltas intermitentes ocorrem em sequência é menor do que a probabilidade de apenas ocorrer $k - 1$ vezes no intervalo de execução. Ao re-executar, espera-se que a falta não ocorra novamente na N-ésima tentativa \cite{DependabilityInEmbeddedSystems} \cite{SchedAndOptOfDistributedFT}.

Portanto, é sacrificado um tempo maior de execução caso a falta ocorra, em troca de um tempo menor de execução médio sem necessitar de componentes extras. Em contraste com a técnica de redundância tripla, é possível entender que a redundância tripla ou "tradicional", depende de uma resiliência espacial "É improvável que uma falta ocorra em vários lugares ao mesmo tempo", enquanto a re-execução depende de uma resiliência temporal: É menos provável que múltiplas faltas ocorram repetidamente em $N$ execuções \cite{FaultTolerantSystems}.

\begin{figure}[H]
    \centering
    \captionsetup{justification=centering}
    \caption{Exemplo de reexecuções}
	\figuraborda{width=1.0\textwidth}{assets/redundancia_reexec.png}
    \captionsetup{justification=raggedright}
    \caption*{Fonte: Elaborada pelo autor}
    \label{fig:redundanciaReexec}
\end{figure}

Na \autoref{fig:redundanciaReexec} é possível observar a reexecução de uma tarefa em duas modalidades: Na primeira é realizado um consenso entre os resultados das execuções, e na segunda a tarefa é apenas reexecutada até $N$ vezes (podendo então tolerar até $N$ faltas transientes), encerrando sua execução caso nenhuma falta seja detectada. A segunda modalidade é particularmente útil para a implementação de condições de transparência \cite{SchedAndOptOfDistributedFT} que serão abordadas posteriormente.

\subsection{Redundância}

Adicionar redundância ao sistema é uma das formas mais intuitivas e mais antigas de aumentar a tolerância à faltas, a probabilidade de $N$ faltas transientes ocorrendo simultaneamente em um sistema é mais baixa do que a probabilidade de apenas 1 falta \cite{SchedAndOptOfDistributedFT}.

Uma técnica de redundância comum é o uso de redundância modular, tipicamente com 3 instâncias replicadas (neste caso chamada de TMR ou "Triple Modular Redundancy"), a \autoref{fig:redundanciaTMR} ilustra a execução concorrente seguida de um votador.

O uso de TMR para uma tarefa pode consistir na reexecução concorrente das três instâncias, tendo seus resultados decidos por um votador. No contexto de tempo-real é importante que caso alguma das tasks no processo de execução concorrente viole seu prazo, o votador ainda deve escolher um resultado para garantir o critério de tempo-real. O uso de TMR é elegante em sua simplicidade e consegue atingir um bom grau de resiliência, porém com o custo adicional de triplicar o custo em termos de memória e execução \cite{DependabilityInEmbeddedSystems}, e potencialmente necessitar de hardware mais poderoso para manter a mesma performance esperada.

\begin{figure}[H]
    \centering
    \captionsetup{justification=centering}
    \caption{Exemplo de execução com redundância}
	\figuraborda{width=1.0\textwidth}{assets/redundancia_tmr.png}
    \captionsetup{justification=raggedright}
    \caption*{Fonte: Elaborada pelo autor}
    \label{fig:redundanciaTMR}
\end{figure}

Sistemas distribuídos, sejam estes embarcados ou não, também podem aproveitar de sua redundância natural para ter maior dependabilidade \cite{MakingReliableDistSystems}. Faltas em um nó podem ser propagadas e no caso de faltas permanentes em um nó, os outros podem suplantar a execução de suas tarefas mantendo a qualidade média de serviço \cite{MakingReliableDistSystems} \cite{SchedAndOptOfDistributedFT}, o uso de sistemas capazes de auto reparo é vital para a existência de telecomunicação em larga escala e computação em nuvem.

\subsection{Loop Unrolling e Function Inlining}

Uma otimização comum que compiladores realizam é desenrolar laços de repetição (Loop Unrolling) com a finalidade de reduzir erros no preditor de desvios da CPU, no contexto de tolerância à faltas, é possível utilizar dessa otimização como uma forma de redundância espacial, ao reduzir a possibilidade de pulos dependentes de um valor, torna-se menos provável um salto baseado em uma versão corrompida do mesmo. O desenrolamento pode também ser feito caso exista um limite superior conhecido no laço durante a compilação. \cite{LoopUnrollingARM}.

Outra transformação comum é o inlining de funções, onde o corpo de uma função é copiado como se o código tivesse sido diretamente escrito em seu ponto de chamada. Ao reduzir a quantidade de pulos é possível melhorar a coerência do cache de instruções além de permitir outras otimizações durante os passes otimizantes do compilador \cite{EngineeringACompiler}, causando uma melhora na performance. No caso de tolerância à faltas, ao reduzir a quantia de jumps e prover redundância de instruções, o inlining pode também reduzir a chance de um salto inadequado \cite{MakingReliableDistSystems}.

Importante ressaltar que aplicar \textit{function inlining} e \textit{loop unrolling} de forma excessiva pode resultar no oposto do que se deseja no quesito de performance, quando aplicadas de forma agressiva, essas otimizações saturam o cache de instruções, utilizam de mais registradores e ocupam espaço desnecessário no executável \cite{EngineeringACompiler}. Portanto, é importante que estas técnicas não sejam aplicadas de forma arbitrária e que seu uso seja acompanhado de medição para confirmar sua efetividade.

\subsection{Injeção de faltas}

Para adequadamente testar a dependabilidade do sistema, é possível deliberadamente causar faltas com o propósito de catalogar e validar se o sistema atinge as métricas necessárias. Dentre os tipos de teste que podem ser realizados, é possível categorizá-los em quatro grupos principais:

Injeção \textbf{Física}: Envolve utilizar um ambiente físico genuíno para causar as faltas, o principal benefício desta técnica é replicar eventos reais que possam causar faltas, assim como poder injetar faltas em superfícies reais do dispositivo \cite{FaultInjectionTechniques}. O principal problema é que esta técnica é particularmente cara e requer auxílio de equipamentos e profissionais especializados, também não é possível injetar um tipo específico de dado para testar um caso específico.

Injeção \textbf{Lógica em Hardware}: Utiliza-se de um dispositivo adicional para injetar as faltas que controla o dispositivo alvo, possui como vantagem ser menos intrusivo e ainda permitir um algo grau de controle e simulação dos fenômenos físicos, desvantagens incluem uma área maior de circuito necessária, implementação de uma unidade extra e criação de canais de comunicação com o dispositivo alvo \cite{FaultInjectionTechniques}.

Injeção \textbf{Lógica em Software}: Funções são executadas em software para injetar faltas em outras partes do programa, o método é pouco invasivo, de baixo custo, alta portabilidade e permite um controle muito elevado sobre os pontos de injeção e estilo de falta \cite{FaultInjectionTechniques}. Possui a desvantagem de aumentar o tempo médio de execução ao introduzir um custo extra de memória para armazenar o código de injeção, e nem sempre reproduz precisamente fenômenos físicos.

Injeção \textbf{Simulada}: O dispositivo é executado em um ambiente totalmente simulado, tem como vantagem não ser invasivo, altamente flexível e nem sequer necessitar de uma versão física do dispositivo, porém tipicamente requer software de simulação potencialmente caro assim como uma descrição do chip na forma de alguma linguagem de descrição de hardware, que raramente é disponibilizada \cite{FaultInjectionTechniques}.


\section{Sistemas embarcados}

Sistemas embarcados são uma família vasta de sistemas computacionais, algumas das principais características de sistemas embarcados são:

\textbf{Especificidade}: Diferente de um sistema de computação mais generalizado como um computador pessoal ou um servidor, sistemas embarcados são especializados para uma solução de escopo restrito. Exemplos de sistemas embarcados variam de microcontroladores encontrados em carros, televisões e dispositivos IoT até sistemas sofisticados de navegação de aeronaves e navios de grande porte \cite{ComputerOrganizationAndDesign}.

\textbf{Limitação de recursos}: Um corolário da natureza especialista destes sistemas, é que recursos alocados para o sistema são definidos previamente. No caso de microcontroladores o poder de processamento e quantidade de memória podem ser restritos para satisfazer uma necessidade de baixo custo de fabricação e menor consumo energético \cite{ComputerOrganizationAndDesign}. Importante notar que existem sistemas embarcados com acesso maior à recursos, como certos equipamentos de rede e aceleradores, mas os recursos do sistema continuam estaticamente delimitados para cumprir sua função específica.

\textbf{Critério Temporal}: Sistemas embarcados, por serem parte de um todo maior, devem realizar sua função com o mínimo de interrupção para a funcionalidade geral do contexto externo \cite{OperatingSystemConcepts}. A importância do tempo de execução de uma tarefa de um sistema pode ser classificada em duas principais categorias: Soft Real-Time , e Hard Real-Time , a distinção entre estas categorias é explicada na seção seguinte.

\subsection{Sistemas Operacionais de Tempo-Real}

Um sistema operacional é um conjunto de software que permite o gerenciamento e interação com os recursos do hardware através de uma camada de abstração. O componente essencial de um sistema operacional é o kernel, que sempre está executando, A função primordial do kernel é viabilizar a coexistência de diversas tarefas no sistema, as quais demandam acesso às capacidades do hardware, notadamente o tempo de processamento da CPU e o espaço de memória. De forma simplificada, o kernel pode ser descrito como a "cola" entre as aplicações e os recursos de hardware \cite{OperatingSystemConcepts}.

Um sistema operacional de tempo real (RTOS) é um tipo de sistema operacional mais especializado, tipicamente pequeno, que possui como característica central cumprir um critério temporal Real-Time , que é dividido em 2 categorias:

\begin{itemize}
    \item \textbf{Soft Real-Time}: Um sistema que garante essa propriedade precisa sempre garantir que tarefas de  maior importância tenham prioridade sobre as de menor importância. Sistemas Soft Real-Time  tipicamente operam na escala de milissegundos, isto é, percepção humana \cite{SchedAndOptOfDistributedFT}. O atraso de uma tarefa em um sistema Soft Real-Time  não é desejável, mas não constitui uma falta. Exemplos: Player de DVD, videogames, kiosks de atendimento automáticos.
    
    \item \textbf{Hard Real-Time}: Precisam garantir as propriedades de Soft Real-Time , além disso, o atraso de uma tarefa de seu prazo, é inaceitável, para um sistema Hard Real-Time  uma resposta com atraso é o mesmo que resposta nenhuma. Cuidado adicional deve ser utilizado ao projetar sistemas Hard Real-Time , pois muitas vezes aparacem em contextos críticos \cite{ModernOperatingSystems}. Exemplos: Software para sistema de frenagem, sistemas de navegação em aplicações aeroespaciais, software de negociação de alta frequência, fila de mensagens de alta performance
\end{itemize}

Como sistemas que cumprem o critério Hard Real-Time também cumprem os requisitos Soft Real-Time, os sistemas operacionais de tempo real tipicamente tem sua arquitetura orientada a serem capazes de cumprir o critério Hard Real-Time  \cite{SchedAndOptOfDistributedFT}.

Diferente de sistemas operacionais focados em uso geral como Windows, Linux e OSX, os RTOS não priorizam fornecer ao usuário uma sensação de fluidez e adaptabilidade. Devido à seus requisitos temporais rígidos, um RTOS é feito com um foco significativo em determinismo, confiabilidade e simplicidade, para garantir que tarefas sejam executadas com um respeito estrito de seus prazos \cite{OperatingSystemConcepts}. Exemplos de RTOS disponíveis no mercado incluem: FreeRTOS, VxWorks, Zephyr e LynxOS.

\subsection{Escalonador}

O escalonador é o componente do sistema operacional responsável por gerenciar múltiplas tarefas que desejam executar \cite{OperatingSystemConcepts}, sendo um componente extramente crucial, a implementação do escalonador deve garantir que tarefas de alta prioridade executem antes e que a a troca de contexto seja o mais rápido possível. O algoritmo de escalonamento é o fator central para o comportamento do escalonador, sendo categorizados em dois princpais grupos:

\begin{itemize}
    \item \textbf{Cooperativos}: Tarefas precisam voluntariamente devolver o controle da CPU (com exceção de certas interrupções de hardware) para que as outras tarefas possam executar, isso pode ser feito explicitamente por uma função de largar ou implicitamente ao utilizar uma rotina assíncrona do sistema, como ler arquivos, receber pacotes de rede ou aguardar um evento \cite{OperatingSystemConcepts}.

    \item \textbf{Preemptivos}: Além de poderem transferir a CPU manualmente, o escalonador forçará trocas de contexto caso uma condição para a troca seja satisfeita. O algoritmo mais comum que serve de base para diversos escalonadores preemptivos é o Round-Robin onde tarefas possuem uma quantia de tempo máximo alocada para sua execução contínua, nomeada "fatia de tempo" ou "quantum" \cite{ModernOperatingSystems}. Tarefas ainda podem possuir relações de prioridade, alterando a ordem que o escalonador realiza seu despache assim como o tamanho de sua fatia de tempo.
\end{itemize}

Sistemas operacionais de tempo real são tipicamente executados no modo totalmente preemptivo, mas o uso cooperativo também é viável e possui a vantagem de possuir um controle mais granular da execução das tarefas, mas é importante que seja tomado o cuidado adequado para que nenhum prazo de execução Hard Real-Time seja violado por uma tarefa inadvertidamente utilizando a CPU por uma quantidade longa de tempo.

\subsection{Concorrência e Assincronia}

Será utilizado a definição de concorrência como a habilidade de um sistema de lidar com múltiplas tarefas computacionais dividindo seus recursos (particularmente tempo de CPU e memória). Isto é, um sistema não necessariamente precisa ser paralelo (execuções múltiplas simultâneas) para possuir concorrência, mas para tornar paralelismo viável, o sistema necessita de mecanismos de concorrência \cite{MakingReliableDistSystems}.

Uma característica central para a utilidade de concorrência mesmo em situações em que paralelismo é limitado ou impossível vai além da pura expressividade do programador, existem assimetrias grandes na velocidade de acesso de disco, memória, rede, e caches da CPU como demonstrado na \autoref{fig:latencia}. Para acessar recursos de forma eficaz, é necessário lidar com suas características inerentemente assíncronas. O uso de concorrência permite que uma tarefa seja suspensa e resumida (voluntariamente ou não) o que permite que o sistema não fique excessivamente ocioso \cite{OperatingSystemConcepts}.

\begin{figure}[H]
    \centering
    \captionsetup{justification=centering}
    \caption{Hierarquia de latência de acessos}
	\figuraborda{width=0.8\textwidth}{assets/latency_pyramid.png}
    \captionsetup{justification=raggedright}
    \caption*{Fonte: Adaptado de \citefigure{ModernOperatingSystems}}
    \label{fig:latencia}
\end{figure}

Implementar os mecanismos de concorrência adequados também permite lidar com interrupções de forma mais estruturada, o problema clássico de lidar com uma interrupção é restaurar a memória de pilha e registradores de forma adequada. Interrupções introduzem um fluxo de programa não local , violando as garantias fortes de escopo e ponto de entrada fornecidas por funções.

É uma tendência atual aumentar o número de núcleos em dispositivos devido à velocidades de relógio das CPUs possuirem ganhos marginais em relação ao impacto térmico. A maioria dos computadores de propósito geral (smartphones, tablets, desktops) tipicamente possuem 2 núcleos ou mais \cite{ComputerOrganizationAndDesign}. Essa tendência não se restringe apenas à computadores gerais, sistemas embarcados comerciais também podem se beneficiar tremendamente das possibilidades de paralelismo providas por mais de um núcleo, porém, é importante ressaltar que o uso de estado compartilhado se torna muito mais sensível à erros em um ambiente com múltiplos fluxos de execução, e medidas devem ser tomadas para evitar condições de corridas e deadlocks \cite{OperatingSystemConcepts}.


\subsection{Alocadores Arena}

Alocadores arena são uma técnica de gerenciamento de memória simples. Este tipo de alocador opera através de um princípio simples: um ponteiro avança linearmente através de um bloco contíguo de memória pré-alocado, marcando a fronteira entre memória utilizada e memória disponível \cite{ArenaAllocation}. Fundamentalmente, um alocador arena pode ser visto como um tipo de espaço de pilha alternativo, ambos são alocados e desalocados com apenas a mudança de um contador, e ambos possuem uma ordem rigidamente FIFO no padrão de alocação válido.

Devido à sua alta velocidade e determinismo, este tipo de alocador pode ser de grande utilidade no desenvolvimento de sistemas embarcados.

Para os propósitos deste trabalho, alocadores arena são particularmente adequados pois o ciclo de vida das tarefas e seus dados associados são intrinsecamente acoplados. Quando uma tarefa é criada, toda sua memória de trabalho pode ser alocada de uma arena, ao final da tarefa a memória pode ser rapidamente recuperada.

\subsection{Execução de Tarefas na Presença de Faltas}

Para uma representação mais clara e eficaz de um fluxo de execução sujeito a faltas, é possível utilizar de grafos resilientes a faltas como um mecanismo de diagramação. Nesta representação, os nós correspondem a tarefas, as quais podem ser executadas na mesma unidade de processamento ou não. As arestas do grafo representam o fluxo de execução, uma aresta não demarcada indica execução incondicional, enquanto arestas demarcadas com notação de mensagem designam uma execução condicional à transmissão de uma mensagem. Mensagens e tarefas representadas por símbolo circular indicam pontos ordinários no grafo, ao passo que símbolos quadrados denotam condições de transparência. \cite{SchedAndOptOfDistributedFT}.

Um grafo de processos tolerantes é faltas é definido como um grafo não ponderado direcionado acíclico com seus nós representando tarefas/processos, arestas representam o fluxo de execução e arestas nomeadas representam fluxo dependente da entrega de mensagens. Será utilizado a notação $P_X (N)$, onde $X$ é o número identificador da tarefa, e $N$ corresponde à sua $N$-ésima re-execução, por exemplo: $P_2 (1)$ indica a primeira execução da tarefa $P_2$, enquanto $P_1 (3)$ indica a terceira reexecução da tarefa $P_1$. Uma notação similar será utilizada para mensagens entre tarefas, $m_X (N)$, mensagens, assim como tarefas, estão sujeitas à faltas e custos adicionais de detecção, mas ao invés de re-execução, mensagens são re-enviadas \cite{SchedFTWithSoftAndHardConstraints}.

Para melhor exemplificar a importância da detecção das faltas, será tomado como exemplo um grafo simples, com apenas 3 tarefas e uma mensagem. O grafo precisará tolerar uma falta transiente. O fluxo "ideal" é demonstrado na \autoref{fig:ftgSimples}.

\begin{figure}[H]
    \centering
    \captionsetup{justification=centering}
    \caption{Grafo com 3 processos e uma mensagem}
	\figuraborda{width=0.120\textwidth}{assets/ftg_simples.png}
    \captionsetup{justification=raggedright}
    \caption*{Fonte: Elaborada pelo autor}
    \label{fig:ftgSimples}
\end{figure}

Ao incluir os diferentes desvios possíveis na presença de apenas uma falta, o grafo de execução da \autoref{fig:ftgExpandido} é obtido. A incidência de faltas no processamento ou passagem de mensagens é indicada com um ícone de raio.

\begin{figure}[H]
    \centering
    \captionsetup{justification=centering}
    \caption{Mesmo grafo, mas tolerante à uma falta transiente}
	\figuraborda{width=0.75\textwidth}{assets/ftg_expandido.png}
    \captionsetup{justification=raggedright}
    \caption*{Fonte: Elaborada pelo autor}
    \label{fig:ftgExpandido}
\end{figure}

Será introduzido transparência na tarefa $P_2$, isto é, será executada com redundância temporal ou modular de tal forma que as tarefas subsequentes pudessem assumir "como se" uma falta nunca tivesse acontecido em $P_2$ após seu prazo ter sido completo, o grafo na \autoref{fig:ftgTransparencia} demonstra a condição de transparência com um ícone retangular.

\begin{figure}[H]
    \centering
    \captionsetup{justification=centering}
    \caption{Introdução de transparência em $P_2$}
	\figuraborda{width=0.40\textwidth}{assets/ftg_transparencia.png}
    \captionsetup{justification=raggedright}
    \caption*{Fonte: Elaborada pelo autor}
    \label{fig:ftgTransparencia}
\end{figure}

Introduzindo apenas um ponto de transparência é possível reduzir significativamente as possibilidades de execução do sistema, isso é particularmente benéfico para escalonadores ou algoritmos de tratamento de faltas baseados em máquinas de estado finito. Um grafo mais compacto é benéfico a previsibilidade do sistema, estabelecendo uma relação forte de pré-conclusão com sucesso ao respeitar o prazo da tarefa transparente \cite{SchedAndOptOfDistributedFT}.

Este exemplo é simples e tolera apenas uma falta transiente, porém processos complexos com múltiplas mensagens entre si causam um aumento exponencial de complexidade, especialmente caso seja necessário tolerar até $k$ faltas transientes.

Pode-se adicionar pontos de transparência através de reexecução ou redundância modular (se prazo conjunto das $N$ tarefas for determinística) para aumentar a confiabilidade e previsibilidade do sistema. Essa transparência não é gratuita: há um troca entre maior confiabilidade no escalonador e menor imprevisibilidade na execução, com o custo de maior uso de CPU e memória. Todos os pontos precisam ser verificados, o que pode aumentar o tempo ocioso dos núcleos e exigir a extensão do prazo da tarefa para permitir reexecuções suficientes.

\section{Trabalhos relacionados} \label{sec:trabRel}

Durante a pesquisa bibliográfica para a fundamentação deste trabalho, foram selecionados três artigos que abordassem temas de tolerância à faltas em software.

\subsection{Reliability Assessment of Arm Cortex-M Processors under Heavy Ions and Emulated Fault Injection}

No trabalho de \citeindirect{ReliabilityArmCortexUnderHeavyIons} utilizam de um sistema COTS e criam um perfil de faltas com exposição a íons pesados assim como injeção artificial de faltas em software para posteriormente realizar uma adição de formas de detecção de faltas para melhorar a confiabilidade do sistema. Foi possível detectar mais da metade das faltas funcionais apenas com técnicas de software, como indicado na figura \autoref{fig:trabRelIonsPesados}.

\begin{figure}[H]
    \centering
    \captionsetup{justification=centering}
    \caption{Análise de resiliência, dividida por categoria}
	\figuraborda{width=0.75\textwidth}{assets/related_works_heavy_ion_reliability.png}
    \captionsetup{justification=raggedright}
    \caption*{Fonte: \citefigure{ReliabilityArmCortexUnderHeavyIons}}
    \label{fig:trabRelIonsPesados}
\end{figure}

Uma outra observação foi que a memória foi duas ordens de magnitude maior em relação ao banco de registradores, indicando que é necessário um foco maior na detecção de faltas na memória \cite{ReliabilityArmCortexUnderHeavyIons}.

\subsection{Técnica de confiabilidade em nível de sistema operacional para a arquitetura RISC-V}

O trabalho de \citeindirect{TecnicaConfiabilidadeRISCV} aborda uma aplicação de TMR (Replicação com $N = 3$) dos processos em um microkernel para a arquitetura RISC-V, para a validação da implementação é utilizado injeção de faltas lógicas em software emuladas com o depurador GDB e o com QEMU como emulador, o sistema é escrito na linguagem Rust, que é capaz de fornecer garantias de segurança de acessos à memória. Foi observado um aumento de confiabilidade do sistema em troca de um custo adicional de memória e tempo de execução.

\subsection{Application-Level Fault Tolerance in Real-Time Embedded System}

No trabalho de \citeindirect{ApplicationLevelFT} são apresentada técnicas de tolerância à faltas em um sistema operacional chamado BOSS, é utilizado uma interface de thread com a implementação de tolerância conformando à interface. O trabalho naturalmente explora o escalonador mas não entra em detalhamento profundo na parte de detecção, mas sim de prover uma biblioteca na forma de classes representando tarefas resilientes \cite{ApplicationLevelFT}. Um caso de estudo de um sistema de filtragem de radar é utilizado como projeto.

O trabalho demonstra também a viabilidade de prover interfaces mais abstratas que ainda sejam capazes de executar em sistemas de recursos restritos. Demonstraram-se resultados favoráveis para uma forma híbrida de tolerância com menor uso de CPU em relação à redundância tripla utilizando de técnicas em software combinado com um par de processadores com auto checagem (PSP).

\subsection{Análise Comparativa dos trabalhos relacionados}

Após comparar os trabalhos, o de natureza mais similar em termos de arquitetura é o de \citeindirect{ApplicationLevelFT} e o mais similar em termos de técnicas de injeção e materiais utilizados a técnica de confiabilidade apresentada em \citeindirect{TecnicaConfiabilidadeRISCV}. O \autoref{tab:trabRel} demonstra as principais diferenças entre os trabalhos.

\begin{quadro}[H]
    \centering
    \caption{Comparação dos trabalhos relacionados}
    \begin{tabular}{|p{0.20\textwidth}|p{0.15\textwidth}|p{0.15\textwidth}|p{0.15\textwidth}|p{0.20\textwidth}|}
        \hline
        \rowcolor[HTML]{C0C0C0}
        \textbf{Trabalho} & \textbf{Sistema} & \textbf{Hardware} & \textbf{Injeção} & \textbf{Técnicas} \\
        \hline

        \citefigure{ReliabilityArmCortexUnderHeavyIons} & Bare Metal, FreeRTOS & CY8CKIT-059 & Física e Lógica em Software & Redundância de Registradores, Deadlines, Redução de Registradores, Asserts \\
        \hline

        \citefigure{TecnicaConfiabilidadeRISCV} & RISC-V Emulado no QEMU & Microkernel de \citeindirect{BasicMicrokernelForRISCV} & Emulada em Software & Redundância Modular, Segurança de memória extra (Borrow checker do Rust) \\
        \hline

        \citefigure{ApplicationLevelFT} & BOSS & Máquinas PowerPC 823 e um PC x86\_64 não especificado & Simulada em Software & Redundância Modular, Deadlines, Rollback/Retry \\
        \hline

        Este Trabalho & FreeRTOS & STM32 Blackpill & Lógica em Software e Hardware & Heartbeat / Deadline Monitor, Asserts, Reexecução e Redundância de Tarefas \\
        \hline

    \end{tabular}
    \label{tab:trabRel}
\end{quadro}

 \clearpage
\chapter{Projeto}
\label{cap:proj}

Este capítulo detalha o projeto desenvolvido para análise das técnicas de tolerância à falhas. São apresentados a visão geral da solução, as premissas adotadas, a metodologia utilizada, requisitos e seu plano de verificação. Também são detalhados os materiais e combinações de técnicas que serão aplicadas.

\section{Visão Geral}

Serão implementados mecanismos de tolerância utilizando do FreeRTOS como base, as técnicas implementadas serão discutidas em maior detalhe na \autoref{subsec:algoritmos}. Para a coleta das métricas de eficácia e custo computacional será utilizado um cenário de injeção de falhas lógicas em hardware utilizando do depurador em chip ST-Link, os detalhes da campanha de injeção de falhas serão abordados na \autoref{subsec:campanhaInjecao}.

A \autoref{fig:visaoGeral} sumariza a relação entre os principais componentes, as técnicas de tolerância serão implementadas como complementos ao FreeRTOS e o processo de injeção é controlado por um computador externo que envia comandos para o depurador em chip com o propósito de simular uma falha.

\begin{figure}[H]
    \centering
    \captionsetup{justification=centering}
    \caption{Principais componentes do projeto}
    \includegraphics[width=1.0\textwidth]{assets/visao_geral.png}
    \captionsetup{justification=raggedright}
    \caption*{Fonte: Elaborada pelo autor}
    \label{fig:visaoGeral}
\end{figure}

\section{Premissas}

Será partido do ponto que ao menos o processador que executa o escalonador terá registradores de controle (ponteiro de pilha, contador de programa, endereço de retorno) que sejam capazes de mascarar falhas. Apesar de ser possível executar os algoritmos reforçados com análise de fluxo do programa e adicionar redundância aos registradores, isso adiciona um grau grande de complexidade que foge do escopo do trabalho. Como mencionado na \autoref{sec:trabRel}, a memória fora do banco de registradores pode ser duas ordens de magnitude mais sensível à eventos disruptivos \cite{ReliabilityArmCortexUnderHeavyIons}.

Durante os testes, será assumido um critério Hard Real-Time sem presença de falhas, isto é, não é aceitável o descumprimento do prazo de execução quando não houver falhas. Na presença de falhas, o prazo também deve ser mantido porém será considerado preferível que falhas sejam detectadas e causem um atraso ao invés de causar uma corrupção silenciosa.

Com o fim de reduzir o tamanho do executável e manter o fluxo de mais previsível não serão utilizados mecanismos de exceção com desenrolamento (unwinding) da pilha. Também não será utilizado de RTTI (Runtime Type Information). Todos os erros devem portanto ser tratados como valores ou como falhas lógicas.

Necessariamente, é preciso também presumir que testes sintéticos possam ao menos aproximar a performance do mundo real, ou ao menos prever o pior caso possível com grau razoável de acurácia. O uso de testes sintéticos não deve ser um substituto para a medição em uma aplicação real, porém, uma bateria de testes com injeção artificial de falhas pode ser utilizada para verificar as tendências e custos relativos introduzidos, mesmo que não necessariamente reflitam as medidas absolutas do produto final.

Portanto, será assumido que os resultados extraídos de injeção de falhas artificiais, apesar de menos condizentes com os valores absolutos de uma aplicação e não sendo substitutos adequados na fase de aprovação de um produto real, são ao menos capazes para realizar uma análise quanto ao custo proporcional introduzido, e devido à sua facilidade de realização e profundidade de inspeção possível, serão priorizados inicialmente neste projeto.

\section{Metodologia}

\subsection{Materiais}

Será utilizada a linguagem C++ com o compilador GCC (ou Clang), o alvo principal do trabalho será um microcontrolador STM32F411CEU6 "BlackPill" 32-bits da arquitetura ARM, como visto na \autoref{fig:stm32Blackpill}.

\begin{figure}[H]
    \centering
    \captionsetup{justification=centering}
    \caption{Diagrama da STM32F411CEU6 ("BlackPill")}
    \includegraphics[width=0.80\textwidth]{assets/stm32_blackpill.png}
    \captionsetup{justification=raggedright}
    \caption*{Fonte: \citefigure{STMBoardProductPage}}
    \label{fig:stm32Blackpill}
\end{figure}

Para a injeção de falhas será utilizado o depurador GDB em conjunto com uma ferramenta de depuração de hardware ST-LINK (\autoref{fig:stLink}), a comunicação do ST-LINK é feita via USB com o computador e via JTAG com o microcontrolador alvo, também será usado em conjunto ferramentas do fabricante como o CubeIDE, CubeMX e CubeCLT.

\begin{figure}[H]
    \centering
    \captionsetup{justification=centering}
    \caption{ST-LINK/V2}
    \includegraphics[width=0.40\textwidth]{assets/st_link.png}
    \captionsetup{justification=raggedright}
    \caption*{Fonte: \citefigure{STLinkProductPage}}
    \label{fig:stLink}
\end{figure}

\begin{figure}[H]
    \centering
    \captionsetup{justification=centering}
    \caption{STMCubeIDE}
    \includegraphics[width=0.65\textwidth]{assets/stmcube_ide.png}
    \captionsetup{justification=raggedright}
    \caption*{Fonte: \citefigure{STMCubeProductPage}}
    \label{fig:stmCubeIDE}
\end{figure}

Durante a fase de desenvolvimento dos algoritmos será utilizado o QEMU juntamente com as ferramentas anteriormente citadas, assim como sanitizadores de memória e condições de corrida (ASan, TSan, UBSan) para auxiliar na detecção de erros mais cedo durante o desenvolvimento.

O sistema operacional de tempo real escolhido foi o FreeRTOS, por ser extensivamente testado e documentado e prover um escalonador totalmente preemptivo com um custo espacial relativamente pequeno, além disso, os contribuidores do FreeRTOS mantém uma lista grande de versões para diferentes arquiteturas e controladores, facilitando drasticamente o trabalho ao não ter que criar uma HAL do zero.

\subsection{Métodos}

Serão utilizadas as seguintes técnicas de tolerância à falhas implementadas em software: CRCs para dados, redundância modular, reexecução, sinal heartbeat e asserts. O detalhamento específico de cada técnica é abordado em maior detalhe na \autoref{subsec:algoritmos}.

Para a criação da análise, serão realizados testes com injeção lógica em hardware utilizando-se do ST-Link em combinação com um computador que emitirá os comandos para injeção via depurador, as falhas serão de natureza transiente e afetarão valores na memória (corrupção silenciosa). A \autoref{fig:injecaoHardwareLogica} detalha de forma mais específica o fluxo de gerar uma falha. As combinações específicas de falhas e técnicas escolhidas são abordadas na \autoref{subsec:campanhaInjecao}.

\begin{figure}[H]
   \centering
   \captionsetup{justification=centering}
   \caption{Injeção lógica em hardware}
   \includegraphics[width=0.85\textwidth]{assets/injecao_hardware.png}
   \captionsetup{justification=raggedright}
  \caption*{Fonte: Elaborada pelo autor}
   \label{fig:injecaoHardwareLogica}
\end{figure}

A coleta de métricas será realizada com os contadores de incremento atômico, o tempo de execução das tarefas, seu espaço de memória utilizado e o número de falhas detectadas será armazenado em uma estrutura que residirá em um segmento de memória que é deliberadamente isento de falhas. Ao fim da execução, será utilizado o depurador para ler estes valores.

Com o objetivo de promover a reutilização de código, será criada uma interface que abstrai noções comuns de tarefa. A interação da interface com o resto do sistema é abordada em maior detalhe na \autoref{subsec:interface}.

\section{Análise de requisitos}
\label{sec:req}

\begin{quadro}[H]
    \centering
    \caption{Requisitos funcionais}
    \begin{tabular}{|p{0.125\textwidth}|p{0.8\textwidth}|}
        \hline
        \rowcolor[HTML]{C0C0C0}
        \textbf{Requisito} & \textbf{Descrição}  \\
        \hline
        
        \textbf{RF01} & Implementação de todos os algoritmos descritos na \autoref{subsec:algoritmos} \\ 
        \hline

        \textbf{RF02} & Configuração do mecanismo de tolerância, prioridade e prazo de execução da tarefa \\
        \hline

        \textbf{RF03} & Cumprimento do prazo estipulado no momento de criação da tarefa caso não exista presença de falhas \\
        \hline

        \textbf{RF04} & Dependabilidade superior à versão do sistema sem técnicas \\
        \hline
        
        \textbf{RF05} & Monitoramento do número de falhas detectadas e violações de prazos  \\
        \hline
    \end{tabular}
    \label{tab:rf}
\end{quadro}

\begin{quadro}[H]
    \centering
    \caption{Requisitos não funcionais}
    \begin{tabular}{|p{0.125\textwidth}|p{0.8\textwidth}|}
        \hline
        \rowcolor[HTML]{C0C0C0}
        \textbf{Requisito} & \textbf{Descrição}  \\
        \hline
        
        \textbf{RNF01} & O consumo de memória deve ser pré determinado em tempo de compilação ou na inicialização do sistema \\
        \hline
        
        \textbf{RNF02} & A interface deve ser construida sobre o escalonador preemptivo do FreeRTOS \\
        \hline

        \textbf{RNF03} & Deve ser compatível com arquitetura ARMv7M ou ARMv8M \\
        \hline

        \textbf{RNF04} & Implementação realizada em C++ (versão 20 ou acima) \\
        \hline
        
        \textbf{RNF05} & Código Fonte disponível sob licença permissíva \\
    \end{tabular}
    \label{tab:rnf}
\end{quadro}

\subsection{Interface} \label{subsec:interface}

Para melhor generalizar o uso das técnicas, utiliza-se de uma abstração da estrutura de tarefa juntamente com um mecanismo de validação de payload via CRC32. A \autoref{fig:messageStruct} demonstra a estrutura de um envelope de dados que inclui um valor de checagem, note que bytes de enchimento necessitam ser ignorados.

\begin{figure}[H]
    \centering
    \captionsetup{justification=centering}
    \caption{Layout de uma mensagem}
    \includegraphics[width=0.50\textwidth]{assets/payload_layout.png}
    \captionsetup{justification=raggedright}
    \caption*{Fonte: Elaborada pelo autor}
    \label{fig:messageStruct}
\end{figure}

A tarefa é um objeto de interface que abstrai parte do estado utilizado pelo RTOS e provê métodos para sua inicialização, término e cancelamento. Uma tarefa possui um espaço de pilha dedicado, e uma V-Table que inclui os métodos providos para sua execução. A identificação da tarefa se dá pelo seu ID, que diretamente mapeia o recurso que encapsula o estado da tarefa no RTOS. O diagrama na \autoref{fig:bddTarefa} demonstra a relação de uma tarefa e os outros componentes do sistema.

\begin{figure}[H]
    \centering
    \captionsetup{justification=centering}
    \caption{Objeto que implementa a interface de Tarefa}
    \includegraphics[width=0.90\textwidth]{assets/task_bdd.png}
    \captionsetup{justification=raggedright}
    \caption*{Fonte: Elaborada pelo autor}
    \label{fig:bddTarefa}
\end{figure}

\subsection{Algoritmos e Técnicas} \label{subsec:algoritmos}

Para a implementação da funcionalidade de tolerância à falhas, algumas das técnicas abordadas no \autoref{cap:fund} serão utilizadas. O detalhamento sobre a implementação será abordado nesta seção.

\subsubsection{CRC: Cyclic Redundancy Check}

Será implementado o CRC-32C, que já é aplicado em sistemas de arquivos como o Btrfs e o ext4, assim como em protocolos de rede como iSCSI e SCTP. Seu Polinômio geradador $P$ é:

\begin{equation}
    \begin{split}
        P = & x^{32} + x^{28} + x^{27} + x^{26} + x^{25} + x^{23} + x^{22} + x^{20} \\
            & + x^{19} + x^{18} + x^{14} + x^{13} + x^{11} + x^{10} + x^{9} + x^{8} + x^{6} + 1
    \end{split}
\end{equation}
\addEquacao{Polinômio CRC-32C}{5}


\subsubsection{Redundância Modular}

Para a aplicação da redundância modular, neste caso a redundância modular tripla, será feito a replicação concorrente da tarefa, cada tarefa possui um espaço de pilha próprio e são escalonadas de forma convencional pelo FreeRTOS. O corpo das tarefas não é replicado, e continua como parte de memória para apenas leitura e execução, um exemplo da relação de réplicas de tarefas executando em relação ao resto do sistema pode ser observado na \autoref{fig:bddTMR}.

\begin{figure}[H]
    \centering
    \captionsetup{justification=centering}
    \caption{Diagrama de bloco de Redundância modular}
    \includegraphics[width=0.950\textwidth]{assets/tmr_bdd.png}
    \captionsetup{justification=raggedright}
    \caption*{Fonte: Elaborada pelo autor}
    \label{fig:bddTMR}
\end{figure}

\subsubsection{Reexecução}

A implementação de tarefas com reexecução é baseada no uso de execuções consecutivas que reutilizam do mesmo espaço de pilha, uma tarefa pode ser sempre reexecutada $N$ vezes, servindo um propósito similar à técnica de redundância, ou executada \textit{até} $N$ vezes, encerrando a execução imediatamente após não encontrar nenhuma falha. Para os propósitos deste trabalho, será utilizada a segunda técnica, pois permite mais oportunidade para o escalonador encaixar trabalho no tempo ocioso, e também por ser um exemplo mais bem estudado na fundamentação teórica deste trabalho por Isosimov et. al.

Assumindo $N = 3$, o diagrama na \autoref{fig:stateReexec} descreve uma máquina de estado finito para a execução de uma tarefa com reexecução, espera-se que o caso médio seja a execução diretamente para um estado correto. Para a criação de uma condição de transparência, o prazo da tarefa deve ser o pior caso possível de $N$ execuções.

\begin{figure}[H]
    \centering
    \captionsetup{justification=centering}
    \caption{Estados de uma reexecução}
    \includegraphics[width=0.925\textwidth]{assets/state_reexec.png}
    \captionsetup{justification=raggedright}
    \caption*{Fonte: Elaborada pelo autor}
    \label{fig:stateReexec}
\end{figure}

\subsubsection{Sinal Heartbeat / Deadline Monitor}

Para a implementação dos sinais de heartbeat será utilizado uma tarefa que servirá como um monitor que associa uma chave (como o ID da tarefa) a um prazo específico. Será utilizado uma escrita atômica de um número e o temporizador do sistema no momento da escrita, se houver uma violação do prazo combinado na criação da tarefa e o prazo apresentado, é considerado que ocorreu uma falha. Este fluxo é visualmente representado na \autoref{fig:heartbeatAdv}.

\begin{figure}[H]
    \centering
    \captionsetup{justification=centering}
    \caption{Sinal Heartbeat}
    \includegraphics[width=0.95\textwidth]{assets/heartbeat_signal.png} %TODO: Imagem melhor
    \captionsetup{justification=raggedright}
    \caption*{Fonte: Elaborada pelo autor}
    \label{fig:heartbeatAdv}
\end{figure}

\subsubsection{Asserts}

Asserts serão utilizados para verificar invariantes, qualquer quebra de contrato de função ou invariante que é coberta com um assert deve resultar em uma falha.

\section{Plano de Verificação}

Para a validação dos algoritmos e técnicas utilizadas(\textbf{RF01}, \textbf{RF02}) serão feitos testes unitários das técnicas isoladas.

A validação da detecção de falhas e vencimento de prazos (\textbf{RF04}, \textbf{RF05}) serão preliminarmente testadas com injeção lógica em software e será validado definitivamente durante o teste com injeção lógica em hardware. Importante notar que a priorização de tarefas e parte dos algoritmos de comunicação entre tarefas já são implementados no FreeRTOS.

Como o produto final do trabalho requer uma análise de resiliência e do custo das técnicas, a seção seguinte aborda a campanha de injeção utilizada, que será aplicada com método lógico em hardware para a análise final.

\subsection{Programa de Teste}

Para testar a viabilidade e o impacto das técnicas implementadas, será utilizado um programa de teste que lerá uma imagem e aplicará um filtro de detecção de bordas Sobel utilizando convolução. O resultado é salvo num arquivo após ser enviado via VirtualCOM. A saída é construída linha por linha. Cada linha possui um prazo hard real-time de 60ms para ser computada.

O programa requer um uso de CPU alto e realiza uma grande quantidade de acessos à memória, o processo de convolução também necessita de valores estáveis de kernel e offsets para calcular o valor de cada pixel. A fundamentação teórica da aplicação destes algoritmos é primariamente extraídos de um livro de processamento digital de sinais por \citeindirect{GuideToDSP}.

A estrutura do programa com CRC é demonstrada na figura \autoref{fig:programaTeste}, caso o check de CRC não seja bem sucedido, a linha é recomputada com a mesma técnica de tolerância mais uma vez. Caso CRC esteja desabilitado, este passo é ignorado.

\begin{figure}[H]
    \centering
    \captionsetup{justification=centering}
    \caption{Fluxo do programa de exemplo}
    \includegraphics[width=0.90\textwidth]{assets/programa_teste.png} \\
    \captionsetup{justification=raggedright}
    \caption*{Fonte: Elaborada pelo autor}
    \label{fig:programaTeste}
\end{figure}

\subsection{Campanha de Injeção de Falhas} \label{subsec:campanhaInjecao}

Para testar a injeção de falhas serão utilizados mecanismos lógicos em software e em hardware com com o auxílio do depurador ST-Link. As falhas serão de natureza transiente e focarão no segmento de memória com leitura e escrita.

A primeira rodada de injeção consiste em em símbolos conhecidos do programa de exemplo (linha do output e imagem resultado) com o objetivo de demonstrar as técnicas em uma situação de falha previsível, onde o valor e posição exata do upset não são conhecidos, mas não ocorre corrupção direta do estado do stack frame.

A segunda rodada consiste em  em execução e causar um upset de $N$ bytes a partir do endereço de uma variável local. O objetivo deste teste é avaliar como as tarefas irão se comportar na presença de uma falha que diretamente corrompe seu stack frame.

As combinações listadas no \autoref{tab:combinacoesTecnicas} visam observar o impacto das duas técnicas que causam alteração significativa no fluxo de execução (TMR e Reexecução) e da técnica de detecção de corrupção CRC. Todos os testes terão acompanhamento de sinal Heartbeat no início e no final para monitorar cumprimento do prazo de execução global assim como uma instância local.

\begin{quadro}[H]
    \centering
    \caption{Combinações de técnicas utilizadas}
    \begin{tabular}{|c|c|c|c|c|}
        \hline
        \rowcolor[HTML]{C0C0C0}
        \textbf{Reexecução} & \textbf{Redundância modular} & \textbf{CRC} & \textbf{Deadline/Heartbeat} & \textbf{Asserts} \\
        \hline
        - & - & - & X & X \\
        \hline
        - & - & X & X & X \\

        \hline
        X & - & - & X & X \\
        \hline
        X & - & X & X & X \\

        \hline
        - & X & - & X & X \\
        \hline
        - & X & X & X & X \\
        \hline
    \end{tabular}
    \label{tab:combinacoesTecnicas}
\end{quadro}
 \clearpage
\chapter{Desenvolvimento}
\label{cap:desenvolvimento}

\section{Estrutura da Implementação}
Esta seção aborda os detalhes técnicos mais importantes para a implementação das técnicas de tolerância à falhas

\subsection{Estratégia de Alocação}


Com exceção de algumas alocações internas do FreeRTOS e da biblioteca do sistema, a maioria das alocações dinâmicas realizadas é feita com o uso de alocadores do tipo arena de memória. Este padrão de alocador comporta-se similarmente à uma pilha, o incremento e decremento de um ponteiro determina a barreira entre memória disponível e alocada. Portanto alocações são rápidas e simples, assim como liberação total ou reset para um estado anterior, a principal desvantagem destes alocadores é não serem capazes de expressar ordems de alocação e liberação granulares muito distintas \cite{ArenaAllocation}.

Pela estrutura do alocador ser simples, a coleta de métricas e injeção de falhas foi simplificada dado que é possível acessar a maioria dos componentes do sistema apenas em termos de seu deslocamenteo em relação à sua arena que é facilmente identificável no depurador ao pesquisar por seu símbolo.

Cada arena serve para encapsular um lifetime conjunto de $N$ alocações, todas as alocações estão vivas, ou todas estão mortas. Para o gerenciamento de tarefas essa técnica mostrou-se satisfatória e fácil de utilizar, pois tarefas e seu estado local são inseparáveis no comportamento normal do sistema, também é possível facilmente recolher memória de tarefas replicadas canceladas durante a aplicação de técnicas de TMR e Reexecução. A estrutura de arena falha em ser adequada no geranciamento de múltiplos objetos com lifetimes distintos, o único caso que isso ocorre na aplicação é na alocação de observadores de deadline, onde o problema é mitigado com o uso de uma lista intrusiva.

\subsection{Tarefas}

A classe \texttt{RawTask} é uma abstração fina que encapsula a interação com o escalonador e fornece mecanismos unificados para gerenciamento do ciclo de vida das tarefas. Esta estrutura foi projetada para oferecer uma interface consistente que abstrai as especificidades da plataforma subjacente.

\begin{quadro}[H]
    \centering
    \caption{Declaração da estrutura \texttt{RawTask}}

    \begin{lstlisting}[language=C++]
struct RawTask {
    RawTaskFunc func = nullptr;
    Arena* arena = nullptr;
    void* args = nullptr;
    DeadlineSlot* deadline = nullptr;
    u32 stack_size = 0;
    u32 args_size = 0;
    u32 id{};
    Atomic<TaskStatus> _status = TaskStatus_Initialized;
    TaskCancelCallback on_cancel = nullptr;
    RawTaskPlatformSpecificData _specific{};

    TaskStatus status();

    void join(CALLER_LOCATION);

    void cancel(CALLER_LOCATION);

    ~RawTask(){}

    bool _platform_init(Arena* a, usize stack_size, RawTaskFunc func, void* args);
    bool _platform_join();
    bool _platform_cancel();
};
\end{lstlisting}
\label{cod:rawTask}
\end{quadro}

Os atributos centrais da classe demonstrados no Quadro \autoref{cod:rawTask} incluem o ponteiro para o corpo da tarefa (\texttt{func}) que é envelopado por uma função \texttt{task\_wrapper} para permitir a inicialização e saída adequada. Os argumentos da função (\texttt{args}), o tamanho da pilha (\texttt{stack\_size}) e o identificador único da tarefa (\texttt{id}). O macro \texttt{CALLER\_LOCATION} é apenas uma expansão para injetar a localização do código que chamou a função utilizando a funcionalidade de \texttt{std::source\_location} do C++ 20 para facilitar diagnóstico de problemas.

O atributo opcional \texttt{deadline} referencia um estrutura \texttt{DeadlineSlot} que serve como um handle para um monitor, a tarefa pode utilizar este monitor para validar seu prazo de execução que é monitorado externamente. Já o callback \texttt{on\_cancel} é uma rotina opcional que pode ser executada na deleção da tarefa, primariamente para garantir a limpeza de certos recursos mesmo no evento em que o cancelamento impeça a execução dos destrutores inseridos pelo compilador.

O estado da tarefa é mantido através do atributo atômico \texttt{status} que indica o ponto geral de execução da tarefa (Não-Inicializado, Inicializado, Executando, Finalizada, Finalizada com Erro). A utilização de operações atômicas neste caso é essencial para execução em ambientes concorrentes e cuidado extra foi tomado na escolha do ordenamento de memória correto para evitar dupla desalocação de recursos. O mecanismo de alocação em arenas auxilia, já que liberação repetida de memória é idempotente.

A classe genérica \texttt{BasicTask} demonstrada no Quadro \autoref{cod:basicTask} incrementa a classe \texttt{RawTask} com conveniência adicional do uso de funções anônimas com grupos de captura e garantia de tipagem de seu retorno. O tipo \texttt{TaskContext} é passado como argumento obrigatório pois permite que a tarefa sendo executada sinalize cancelamento ou conclusão prematura, permitindo um controle superior no caso de tarefas redundantes.

\begin{quadro}[H]
    \centering
    \caption{Declaração da estrutura \texttt{BasicTask} (Implementação omitida)}

    \begin{lstlisting}[language=C++]
template<
    typename Output,
    Callable<Output, TaskContext> TaskFunc,
    Callable<void, TaskContext> OnCancel
>
struct BasicTask {
    RawTask _task;
    TaskFunc _func;
    OnCancel _on_cancel;
    Option<Output> _result;

    static void _basic_task_wrapper(RawTask* t);

    static void _basic_task_cancel_wrapper(RawTask* t);

    Option<Output> result();

    bool has_result() const;

    TaskStatus status() const;

    RawTask* raw_task();

    u32 id();

    void join();

    void cancel();
};
\end{lstlisting}
\label{cod:basicTask}
\end{quadro}

Apesar da implementação mais complexa o uso desta classe é relativamente ergonômico, sendo similar em seu uso à classe \texttt{std::thread} da introduzida no C++11 como demonstrado no Quadro \autoref{cod:basicTaskExample}.

\begin{quadro}[H]
\centering
\caption{Utilizando uma \texttt{BasicTask} para calcular uma soma}
\begin{lstlisting}[language=C++]
auto numbers = Array<i32, 7>{-4, -2, 0, 6, 9, 2, 1};
auto task = make_basic_task(&arena, [numbers](TaskContext* ctx) -> i32 {
    i32 acc = 0;
    for(i32 x : numbers){
        acc += x;
    }
    return x;
});

task.join();
auto sum = task.result().unwrap();
printf("Sum = %d\n", sum);
\end{lstlisting}
\label{cod:basicTaskExample}
\end{quadro}

\subsection{Deadlines e Heartbeat}

A classe \texttt{DeadlineWatcher} descrita no Quadro \autoref{cod:deadlineWatcher} implementa um vigilante responsável pelo gerenciamento centralizado de deadlines ativas, também podendo ser utilizado como um sinal de heartbeat ao utilizar uma deadline deliberadamente mais lenta para utilizar como uma flag de progresso períodico. Internamente, baseia-se em um vetor de descritores que mantém informações temporais de todas as tarefas sob monitoramento. Um spinlock é utilizado para sincronização pois as seções críticas são pequenas.

É possível também disparar uma tarefa que executa um monitor que é monitorado por outro monitor, permitindo a criação de sub-deadlines arbitrariamente aninhadas, este padrão é uma versão rudimentar da técnica de Árvores de Supervisão que é comumente encontrada em sistemas distribuídos com tolerância à falhas \cite{MakingReliableDistSystems}.

\begin{quadro}[H]
    \centering
    \caption{Declaração da estrutura \texttt{DeadlineWatcher}}

    \begin{lstlisting}[language=C++]
struct DeadlineWatcher {
    Slice<DeadlineSlot> slots;
    Spinlock _lock{};
    Atomic<u32> _count;

    usize count() const;

    bool watch(RawTask* task, Duration limit);

    void remove(DeadlineSlot* node);

    void clear();

    bool scan();

    auto lock_guard(){
        return _lock.guard();
    }
};

struct DeadlineSlot {
    TimeTick last_tick;
    Duration limit;
    RawTask* task;

    void reset(){ last_tick = tick_now(); }
};
\end{lstlisting}
\label{cod:deadlineWatcher}
\end{quadro}

Para a leitura e comparação de temporizadores, são utilizadas 2 funções específicas para cada plataforma: \texttt{tick\_now} para observar o estado temporal atual e \texttt{tick\_frequency} para ler a frequência do temporizador.

A funcionalidade de \texttt{scan} é responsável por validar as deadlines e limpar descritores que foram encerrados, o seu retorno indica um estado de sucesso. No caso do uso do temporizador embutido da blackpill para watchdog independente (IWDG), é possível utilizar o output do monitor principal para decidir se o watchdog deve ser reiniciado ou não.

\subsection{CRC32}

A verificação de integridade via CRC32 utiliza a técnica de LUT (Look-up table) para todos os possíveis bytes de entrada considerando o polinômio gerador escolhido, armazenando-os em uma tabela estática. Durante o cálculo do CRC cada byte dos dados é processado através de uma operação de indexação na tabela seguida de uma operação XOR com o estado acumulado, eliminando a necessidade de cálculos bit a bit. O valor final também pode ser incrementalmente computado, que é essencial para estruturas que possuem descontinuidades em memória ou que sejam de tamanho muito grande.

Cuidado adicional deve ser tomado ao calcular o CRC de uma estrutura arbitrária, pois compiladores frequentemente adicionam bytes de enchimento (padding) para garatir alinhamento adequado dos dados na memória. O conteúdo destes bytes é indefinido, portanto calcular o CRC de duas estruturas idênticas pode resultar em números diferentes devido à inclusão de bytes de enchimento com valores distintos. Para remediar isso, uma restrição de tipo \texttt{CRC\_Checkable} deve ser implementada por todas as estruturas que desejam fornecer essa funcionalidade, é parte do contrato do conceito que o implementador adequadamente lide com bytes de enchimento.

\subsection{Asserts}

Foram utilizados 2 níveis de assert, globais (críticos) e locais (contextuais). Para evitar conflito com o macro da biblioteca padrão, a função de assert é nomeada com o sinônimo \texttt{ensure}. A função \texttt{ensure} global verifica um predicado, caso seja falso, dispara uma exceção do processador para causar uma reinicialização forçada, indicando uma falha catastrófica no sistema. A função membro \texttt{TaskContext::ensure} não causa uma falha catastrófica, mas apenas cancela a tarefa.

Asserções locais são utilizadas para detectar a violação de invariantes dentro de um contexto recuperável, já asserções globais são reservadas para propriedades fundamentais que não podem ser violadas.

\subsection{Dependências Adicionais}

A camada de abstração de hardware é fornecida pela STMicroelectronics através de sua ferramenta de geração de código STMCubeMX. Para a comunicação VirtualCOM por USB foi utilizado o pacote de software USB\_DEVICE da STMicroelectronics. Uma biblioteca utilitária geral (\texttt{base}) que é compatível com alvos freestanding é fornecida pelo autor. A biblioteca de domínio público \texttt{stb\_sprintf} por Sean Barrett foi usada para permitir uma implementação consistente de formatação textual entre todas as plataformas, apesar de não ser parte análise do trabalho, foi extramemente útil para diagnosticar e reportar dados do sistema durante o desenvolvimento.

\subsection{Compilação do Projeto}

Apesar do alvo de compilação do projeto ser um microcontrolador com o FreeRTOS com CMSISv2, o design transparente permitiu também a implementação para testes e um ciclo de debug mais rápido em sistemas operacionais completos (Linux e Windows 10/11) sem necessitar de um simulador. Os detalhes de plataforma são consolidados em uma única unidade de tradução e condicionalmente compilados dependendo do sistema alvo. Todo código específico de plataforma pode ser encontrado em suas respectivas unidades \texttt{platform\_PLATAFORMA.cpp}.

\subsection{Programa de Teste}

O programa de teste escolhido foi desenvolvido em 3 versões (Controle, TMR, Reexecução), o algoritmo de convolução é o mesmo como demonstrado no \autoref{cap:proj}, tendo como alteração apenas o setup das tarefas adicionais (no caso de TMR) e a alocação inicial de memória. Para carregar a armazenar a imagem embutida, foi escrito um pequeno carregador de imagem PGM do tipo binário (P5).
 \clearpage
\chapter{Resultados}
\label{cap:resultados}

\section{Dependabilidade e Performance}
Após repetir o processo de injeção do capítulo anterior para as combinações de técnicas, foi possível coletar os seguntes dados

\subsection{Impacto na Detecção}

% TODO Falaar aq

| Injeção | Técnicas | Detectado? | Resultado |
% TODO dados

\subsection{Impacto no Tempo de execução}

É possível observar que não houve uma mudança significativa no tempo de execução entre TMR e reexecução, isso é esperado pois não foi utilizado multi processamento simétrico na placa. Indicando que as diferenças possam ser variações parcialmente causadas pelas trocas de contexto e passagem nas filas.

| Injeção | Técnicas | Tempo de execução (total) | Tempo de execução Médio (ms/Linha) |

% TODO dados

\subsection{Impacto no Uso de Memória RAM}

O maior impacto na memória naturalmente será a presença de novas tasks, portanto a execução TMR gera o maior impacto. O impacto na memória das variáveis globais não possuiu muita variação

| Injeção | Técnicas | Memória de Tarefa (pico) | Memória Compartilhada (pico) |
% TODO dados


% TODO botar isso num apendice?
\subsection{Impacto no Output}

Nem todas as execuções geram o output válido, e algumas geraram o output correto, esta seção serve para cunho de exemplo, demonstrando alguns dos efeitos observados de corrupções silenciosas na imagem resultado.

% TODO: Lena OG e Lena Filtrada (baseline)
% TODO: Lena com kernel corrompido
% TODO: Lena com tamanho corrompido (pular pedaço da imagem)
% TODO: Lena interrompida, filtro parcial

\section{Verificação}

Após a análise das técnicas, é possível verificar os requisitos funcionais na \autoref{tab:verirf} e não-funcionais no \autoref{tab:verirnf}.

\begin{quadro}[H]
    \centering
    \caption{Requisitos funcionais}
    \begin{tabular}{|p{0.125\textwidth}|p{0.4\textwidth}|p{0.4\textwidth}|}
        \hline
        \rowcolor[HTML]{C0C0C0}
        \textbf{Requisito} & \textbf{Descrição} & \textbf{Validação} \\
        \hline

        \textbf{RF01} & Implementação de todos os algoritmos descritos na \autoref{subsec:algoritmos} & TODO \\
        \hline

        \textbf{RF02} & Configuração do mecanismo de tolerância, prioridade e prazo de execução da tarefa & TODO\\
        \hline

        \textbf{RF03} & Cumprimento do prazo estipulado no momento de criação da tarefa caso não exista presença de falhas & Validado com os testes sem injeção de falha \\
        \hline

        \textbf{RF04} & Dependabilidade superior à versão do sistema sem técnicas & O sistema foi capaz de mascarar algumas falhas e detectar outras. Portanto possui uma dependabilidade superior \\
        \hline

        \textbf{RF05} & Monitoramento do número de falhas detectadas e violações de prazos & Implementado em \texttt{DeadlineWatcher} em conjunto com o IWDG. Foi capaz de detectar violações \\
        \hline
    \end{tabular}
    \label{tab:verirf}
\end{quadro}

Os requisitos não funcionais também foram analisados conforme o \autoref{tab:verirnf}.

\begin{quadro}[H]
    \centering
    \caption{Validação dos Requisitos não funcionais}
    \begin{tabular}{|p{0.125\textwidth}|p{0.4\textwidth}|p{0.4\textwidth}|}
        \hline
        \rowcolor[HTML]{C0C0C0}
        \textbf{Requisito} & \textbf{Descrição} & \textbf{Validação} \\
        \hline

        \textbf{RNF01} & O consumo de memória deve ser pré determinado em tempo de compilação ou na inicialização do sistema & O pico de memória é estável e determinado no startup da aplicação \\
        \hline

        \textbf{RNF02} & A interface deve ser construida sobre o escalonador preemptivo do FreeRTOS & Os mecanismos de tarefa são implementados sobre o escalonador \\
        \hline

        \textbf{RNF03} & Deve ser compatível com arquitetura ARMv7M ou ARMv8M & O projeto compila e executa na arquitetura \\
        \hline

        \textbf{RNF04} & Implementação realizada em C++ (versão 20 ou acima) & O projeto compila na versão C++20 com GCC e Clang\\
        \hline

        \textbf{RNF05} & Código Fonte disponível sob licença permissíva & Validado parcialmente. O módulo de USB\_DEVICE não é compatível com uma licença BSD2 porém não é essencial apra a demonstração das técnicas. O resto do código cumpre este requisito. \\
        \hline
    \end{tabular}
    \label{tab:verirnf}
\end{quadro}

% TODO: Link do github com historico limpo

 \clearpage
\chapter{Considerações Finais}
\label{cap:consideracoes}

Sistemas embarcados são sistemas computacionais com propósito específico e são tipicamente parte da qualidade de serviço de um todo. Tendo essas características em mente, é importante que o custo dos recursos necessários para a execução correta de um projeto sejam considerados para a viabilidade do produto final com o máximo de funcionalidade possível com o mínimo de custo de produção e custo energético.

O uso de sistemas operacionais de tempo real em sistemas embarcados é popular por fornecer garantias em relação à prioridade de tarefas assim como permitir um modelo de aplicação assíncrono orientado à eventos.

A dependabilidade é uma propriedade geral do sistema que combina diversos critérios (RAMS) e o aumento da dependabilidade em sistemas de natureza crítica é importante para a segurança dos usuários, outros sistemas também podem se beneficiar ao ofertar uma melhor qualidade de serviço mesmo na presença de adversidades.

Para melhorar a propriedade de dependabilidade de um sistema embarcado, foram exploradas técnicas de execução, monitoramento de prazo e de checagem de integridade, em cenários de falhas transientes.

% Foi observado que o monitoramento de prazos de execução foi importante para a detecção de violações de prazo e foi de grande ajuda no desenvolvimento para detectar a presença de erros lógicos que ocasionaram deadlocks.

% TODO

Para elaboração de trabalhos futuros, é importante analisar o impacto da presença de múltiplos núcleos em um sistema com multi processamento simétrico, assim como expandir a detecção de erros para incluir mecanismos de análise de fluxo de controle a nível de compilador.
 \clearpage

\postextual
\bibliography{refs.bib}  \clearpage

\end{document}

% TODO: Renomear todas as intancias de blackpill pra bluepill
% TODO: Mostrar o programa de convolução como programa exemplo
% TODO: Mostrar injeção de falhas com PyOCD


