\chapter{Resultados}
\label{cap:resultados}

\section{Dependabilidade e Performance}
Após repetir o processo de injeção do capítulo anterior para as combinações de técnicas, foi possível coletar os seguntes dados

\subsection{Impacto na Detecção}

% TODO Falaar aq

| Injeção | Técnicas | Detectado? | Resultado |
% TODO dados

\subsection{Impacto no Tempo de execução}

É possível observar que não houve uma mudança significativa no tempo de execução entre TMR e reexecução, isso é esperado pois não foi utilizado multi processamento simétrico na placa. Indicando que as diferenças possam ser variações parcialmente causadas pelas trocas de contexto e passagem nas filas.

| Injeção | Técnicas | Tempo de execução (total) | Tempo de execução por pixel |
% TODO dados

\subsection{Impacto no Uso de Memória RAM}

O maior impacto na memória naturalmente será a presença de novas tasks, portanto a execução TMR gera o maior impacto. O impacto na memória das variáveis globais não possuiu muita variação

| Injeção | Técnicas | Memória de Tarefa (pico) | Memória Compartilhada (pico) |
% TODO dados


% TODO botar isso num apendice?
\subsection{Impacto no Output}

Nem todas as execuções geram o output válido, e algumas geraram o output correto, esta seção serve para cunho de exemplo, demonstrando alguns dos efeitos observados de corrupções silenciosas na imagem resultado.

% TODO: Lena OG e Lena Filtrada (baseline)
% TODO: Lena com kernel corrompido
% TODO: Lena com tamanho corrompido (pular pedaço da imagem)
% TODO: Lena interrompida, filtro parcial

\section{Verificação}

Após a análise das técnicas, é possível verificar os requisitos funcionais na tabela %TODO TAB
e não-funcionais na tabela %TODO TAB 
do projeto da sequinte forma:

% RF
% RnF
