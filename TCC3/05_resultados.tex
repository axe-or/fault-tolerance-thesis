\chapter{Resultados}
\label{cap:resultados}

\section{Dependabilidade e Performance}
Após repetir o processo de injeção do capítulo anterior para as combinações de técnicas, foi possível coletar os seguntes dados

\subsection{Impacto na Detecção}

% TODO Falaar aq

| Injeção | Técnicas | Detectado? | Resultado |
% TODO dados

\subsection{Impacto no Tempo de execução}

É possível observar que não houve uma mudança significativa no tempo de execução entre TMR e reexecução, isso é esperado pois não foi utilizado multi processamento simétrico na placa. Indicando que as diferenças possam ser variações parcialmente causadas pelas trocas de contexto e passagem nas filas.

| Injeção | Técnicas | Tempo de execução (total) | Tempo de execução Médio (ms/Linha) |

% TODO dados

\subsection{Impacto no Uso de Memória RAM}

O maior impacto na memória naturalmente será a presença de novas tasks, portanto a execução TMR gera o maior impacto. O impacto na memória das variáveis globais não possuiu muita variação

| Injeção | Técnicas | Memória de Tarefa (pico) | Memória Compartilhada (pico) |
% TODO dados


% TODO botar isso num apendice?
\subsection{Impacto no Output}

Nem todas as execuções geram o output válido, e algumas geraram o output correto, esta seção serve para cunho de exemplo, demonstrando alguns dos efeitos observados de corrupções silenciosas na imagem resultado.

% TODO: Lena OG e Lena Filtrada (baseline)
% TODO: Lena com kernel corrompido
% TODO: Lena com tamanho corrompido (pular pedaço da imagem)
% TODO: Lena interrompida, filtro parcial

\section{Verificação}

Após a análise das técnicas, é possível verificar os requisitos funcionais na \autoref{tab:verirf} e não-funcionais no \autoref{tab:verirnf}.

\begin{quadro}[H]
    \centering
    \caption{Requisitos funcionais}
    \begin{tabular}{|p{0.125\textwidth}|p{0.4\textwidth}|p{0.4\textwidth}|}
        \hline
        \rowcolor[HTML]{C0C0C0}
        \textbf{Requisito} & \textbf{Descrição} & \textbf{Validação} \\
        \hline

        \textbf{RF01} & Implementação de todos os algoritmos descritos na \autoref{subsec:algoritmos} & TODO \\
        \hline

        \textbf{RF02} & Configuração do mecanismo de tolerância, prioridade e prazo de execução da tarefa & TODO\\
        \hline

        \textbf{RF03} & Cumprimento do prazo estipulado no momento de criação da tarefa caso não exista presença de falhas & Validado com os testes sem injeção de falha \\
        \hline

        \textbf{RF04} & Dependabilidade superior à versão do sistema sem técnicas & O sistema foi capaz de mascarar algumas falhas e detectar outras. Portanto possui uma dependabilidade superior \\
        \hline

        \textbf{RF05} & Monitoramento do número de falhas detectadas e violações de prazos & Implementado em \texttt{DeadlineWatcher} em conjunto com o IWDG. Foi capaz de detectar violações \\
        \hline
    \end{tabular}
    \label{tab:verirf}
\end{quadro}

Os requisitos não funcionais também foram analisados conforme o \autoref{tab:verirnf}.

\begin{quadro}[H]
    \centering
    \caption{Validação dos Requisitos não funcionais}
    \begin{tabular}{|p{0.125\textwidth}|p{0.4\textwidth}|p{0.4\textwidth}|}
        \hline
        \rowcolor[HTML]{C0C0C0}
        \textbf{Requisito} & \textbf{Descrição} & \textbf{Validação} \\
        \hline

        \textbf{RNF01} & O consumo de memória deve ser pré determinado em tempo de compilação ou na inicialização do sistema & O pico de memória é estável e determinado no startup da aplicação \\
        \hline

        \textbf{RNF02} & A interface deve ser construida sobre o escalonador preemptivo do FreeRTOS & Os mecanismos de tarefa são implementados sobre o escalonador \\
        \hline

        \textbf{RNF03} & Deve ser compatível com arquitetura ARMv7M ou ARMv8M & O projeto compila e executa na arquitetura \\
        \hline

        \textbf{RNF04} & Implementação realizada em C++ (versão 20 ou acima) & O projeto compila na versão C++20 com GCC e Clang\\
        \hline

        \textbf{RNF05} & Código Fonte disponível sob licença permissíva & Validado parcialmente. O módulo de USB\_DEVICE não é compatível com uma licença BSD2 porém não é essencial apra a demonstração das técnicas. O resto do código cumpre este requisito. \\
        \hline
    \end{tabular}
    \label{tab:verirnf}
\end{quadro}

% TODO: Link do github com historico limpo

