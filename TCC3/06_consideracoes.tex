\chapter{Considerações Finais}
\label{cap:consideracoes}

Sistemas embarcados são sistemas computacionais com propósito específico e são tipicamente parte da qualidade de serviço de um todo. Tendo essas características em mente, é importante que o custo dos recursos necessários para a execução correta de um projeto sejam considerados para a viabilidade do produto final com o máximo de funcionalidade possível com o mínimo de custo de produção e custo energético.

O uso de sistemas operacionais de tempo real em sistemas embarcados é popular por fornecer garantias em relação à prioridade de tarefas assim como permitir um modelo de aplicação assíncrono orientado à eventos.

A dependabilidade é uma propriedade geral do sistema que combina diversos critérios (RAMS) e o aumento da dependabilidade em sistemas de natureza crítica é importante para a segurança dos usuários, outros sistemas também podem se beneficiar ao ofertar uma melhor qualidade de serviço mesmo na presença de adversidades.

Para melhorar a propriedade de dependabilidade de um sistema embarcado, foram exploradas técnicas de execução, monitoramento de prazo e de checagem de integridade, em cenários de falhas transientes.

% Foi observado que o monitoramento de prazos de execução foi importante para a detecção de violações de prazo e foi de grande ajuda no desenvolvimento para detectar a presença de erros lógicos que ocasionaram deadlocks.

% TODO

Para elaboração de trabalhos futuros, é importante analisar o impacto da presença de múltiplos núcleos em um sistema com multi processamento simétrico, assim como expandir a detecção de erros para incluir mecanismos de análise de fluxo de controle a nível de compilador.
