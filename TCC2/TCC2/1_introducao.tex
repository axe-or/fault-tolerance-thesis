\chapter{Introdução}
\label{cap:intro}

Tolerância à falhas (TF) é a capacidade de um sistema computacional continuar oferecendo qualidade de serviço mesmo na presença de defeitos e interferências inesperadas, falhas podem ser encontradas comumente nos contextos de sistemas distribuídos, onde os canais de comunicação podem sofrer degradação ou total inoperabilidade devido à interferência eletromagnética, falta de energia e eventos climáticos \cite{FaultTolerantSystems}. Sistemas embarcados encontram os mesmos problemas ao utilizar um canal com ruído ou instável, mas também podem dados em memória ou registradores diretamente afetados por causas externas como radiação ionizante, flutuações súbitas de temperatura ou voltagem e colisões físicas \cite{DependabilityInEmbeddedSystems}. A existência desses fenômenos necessitam que dispositivos estejam preparados, especialmente em aplicações aeroespaciais que necessitam operam em um ambiente volátil com consequências catastróficas caso um erro ocorra.

Tornar um sistema tolerante à falhas é um problema multi facetado, ambas soluções em hardware e software necessitam ser abordadas para garantir a qualidade de serviço desejada. O escalonador é crucial na execução concorrente de diversas tarefas, sendo então um candidato interessante para aprimorar sua resiliência com foco em reduzir desperdício dos nós computacionais \cite{OperatingSystemConcepts}. O processo de detecção das falhas e seu impacto no grafo de execução assim como nas métricas quantitativas de tempo de CPU e uso de memória portanto deve ser considerado, particularmente no contexto de escalonamento, pois a reação rápida e correta às falhas requer previamente a detecção e elaboração das rotinas de escalonamento de forma adequada \cite{DependabilityInEmbeddedSystems}.

Este trabalho visa portanto fazer uma análise do impacto de diferentes técnicas de detecção durante o escalonamento de tarefas e seu impacto na performance e no fluxo de execução do sistema, para que se possa melhor compreender e evidenciar os custos e benefícios ao tornar um sistema mais resiliente. 

\section{Problematização}

Dada a presença de sistemas embarcados em contextos críticos como na exploração espacial, automobilística e tecnologia médica, assim como a ubiquidade de dispositivos móveis e de baixo consumo energético no mercado consumidor (Celulares, Notebooks, equipamentos IoT) e a existência do mercado de computação em nuvem e computação distribuída, entende-se que manter um alto grau da qualidade de serviço com o mínimo de degradação de performance e aumento de custo (monetário ou energético), pode prover uma vantagem econômica para fabricantes e provedores assim como um benefício social na maior confiabilidade no caso de aplicações críticas. \cite{DependabilityInEmbeddedSystems}.

Ademais, ocorreu nos últimos anos uma maior adoção de sistemas COTS (Commercial off the shelf), dado que estes sistemas podem ser mais baratos e fornecem uma solução "genérica" para problemas que anteriormente necessitariam de hardware com um design mais especializado \cite{CyberSecSpaceCOTS}. Mesmo no caso em que se deseja utilizar um design especializado para o produto final, estes sistemas são excelentes para a fase de prototipação e validação do projeto, dado sua facilidade de acesso e flexibilidade.

Um outro fator que influencia na adoção do uso de COTS para certas aplicações que necessitam de tolerância à falhas são as regulações ITAR (International Traffic in Arms Regulations) imposta pelos Estados Unidos que restringe a exportação de diversos tipos de tecnologia de cunho potencialmente militar. Neste caso, o impedimento da exportação certos tipos  tecnologias de PCBs e firmware aumenta mais ainda a necessidade de compradores de outros países adquirirem alternativas comerciais mais comuns que já são complacentes com a regulação \cite{ITARPCBCompliance}.

O custo de utilizar técnicas de tolerância é sensível ao contexto da aplicação e ao nível de tolerância desejado, e no caso dos sistemas COTS, técnicas robustas de resiliência em hardware nem sempre estão disponíveis, sendo necessário delegar tal funcionalidade para a aplicação. Dentre uma multitude de técnicas de detecção e reação à falhas, é necessário escolher a mais adequada, e para que a escolha seja informada, é essencial que os tradeoffs em termos de performance e grau de confiabilidade adquiridos sejam compreendidos, para que se minimize o custo relativo para manter uma qualidade de serviço desejada.

\subsection{Solução Proposta}

Implementar e comparar técnicas de escalonamento com detecção de erros, com o objetivo de esclarecer o impacto de performance em relação ao ganho de dependabilidade do sistema, particularmente no contexto de sistemas com restrição \textit{Hard Real Time}, pois se uma técnica é capaz de satisfazer o critério de tempo real mais rígido, também poderá ser usada em contextos com critério temporal mais relaxado.

\section{Objetivos}

Esta seção formaliza os objetivos do trabalho, conforme descrito a seguir.

\bigskip
\subsection{Objetivo Geral}

% TODO: Objetivo geral

\bigskip
\subsection{Objetivos Específicos}

\begin{enumerate}
    \item TODO: Objetivos
\end{enumerate}

\section{Metodologia}

O objetivo do trabalho é descritivo e exploratório, as métricas coletadas são de caráter quantitativo e conclusões e observações derivadas do trabalho serão realizadas de maneira indutiva baseadas nas métricas de performance coletadas e comparadas.

Foi realizado uma pesquisa bibliográfica para a fundamentação e escolha das técnicas e dos materiais do trabalho, sendo esta primariamente focada em autores com obras associadas ao tema de tolerância assim como temas adjacentes relevantes como sistemas operacionais e interface hardware-software.

Após isso será realizado uma implementação e testes das técnicas escolhidas para validação, e uma campanha de injeção de falhas será realizada em um microcontrolador para a coleta final das métricas, este assunto será aprofundado no \autoref{cap:proj}.


\section{Estrutura do Trabalho}

No \autoref{cap:fund} serão explorados os tópicos centrais que constituem a premissa do trabalho, primariamente conceitos de detecção de falhas e escalonamento nos sistemas operacionais. Sistemas operacionais são um tópico particularmente vasto, portanto sua abordagem será focada apenas nos aspectos mais essenciais para o trabalho. No \autoref{cap:proj} é definido o projeto e o plano de verificação para a implementação e análise das técnicas descritas. No \autoref{cap:consid} é realizado uma recapitulação do que foi abordado no trabalho.


