\chapter{Introdução}
\label{cap:intro}

A tolerância a falhas (TF) refere-se à capacidade de um sistema computacional de manter a qualidade do serviço mesmo diante de defeitos e interferências inesperadas. As falhas são comumente observadas em sistemas distribuídos, onde os canais de comunicação estão sujeitos a degradação ou inoperabilidade total devido à interferência eletromagnética, falta de energia e eventos climáticos \cite{FaultTolerantSystems}. Sistemas embarcados enfrentam problemas semelhantes ao operar com canais sujeitos a ruído ou instabilidade, mas também experimentam dados em memória ou registradores sendo diretamente afetados por causas externas, como radiação ionizante, flutuações repentinas de temperatura ou voltagem, e colisões físicas \cite{DependabilityInEmbeddedSystems}. A presença desses fenômenos exige que os dispositivos estejam adequadamente preparados, especialmente em aplicações aeroespaciais, que operam em ambientes voláteis e enfrentam consequências potencialmente catastróficas caso ocorra um erro.

Tornar um sistema tolerante à falhas é um problema multi facetado, ambas soluções em hardware e software necessitam ser abordadas para garantir a qualidade de serviço desejada. O escalonador é crucial na execução concorrente de diversas tarefas, sendo então um candidato interessante para aprimorar sua resiliência com foco em reduzir desperdício dos nós computacionais \cite{OperatingSystemConcepts}. O processo de detecção das falhas e seu impacto no grafo de execução assim como nas métricas quantitativas de tempo de CPU e uso de memória portanto deve ser considerado, particularmente no contexto de escalonamento, pois a reação rápida e correta às falhas requer previamente a detecção e elaboração das rotinas de escalonamento de forma adequada \cite{DependabilityInEmbeddedSystems}.

Este trabalho visa portanto fazer uma análise do impacto de diferentes técnicas de detecção durante o escalonamento de tarefas e seu impacto na performance e no fluxo de execução do sistema, para que se possa melhor compreender e evidenciar os custos e benefícios ao tornar um sistema mais resiliente. 

\section{Problematização}

Sistemas embarcados estão presentes em diversos contextos como:  exploração espacial, indústria automotiva, tecnologia médica, dispositivos de baixo consumo energético e sistemas distribuídos no contexto IoT. Portanto, entende-se que manter alto grau da qualidade de serviço com o mínimo de degradação de performance e aumento de custo (monetário ou energético), pode prover uma vantagem econômica para fabricantes e provedores assim como um benefício social na maior confiabilidade no caso de aplicações críticas. \cite{DependabilityInEmbeddedSystems}.

Ademais, ocorreu nos últimos anos uma maior adoção de sistemas COTS (Commercial off the shelf), por poderem ser economicamente mais viáveis e fornecerem uma solução "genérica" para problemas que anteriormente necessitariam de hardware design especializado \cite{CyberSecSpaceCOTS}. E mesmo caso pretenda-se utilizar um design especializado para o produto, estes sistemas são excelentes para a fase de prototipação e validação do projeto, dado sua facilidade de acesso e flexibilidade.

Um outro fator que influencia na adoção do uso de COTS para certas aplicações que necessitam de tolerância à falhas são as regulações ITAR (International Traffic in Arms Regulations) imposta pelos Estados Unidos que restringe a exportação de diversos tipos de tecnologia de cunho potencialmente militar. Neste caso, o impedimento da exportação certos tipos  tecnologias de PCBs e firmware aumenta mais ainda a necessidade de compradores de outros países adquirirem alternativas comerciais mais comuns que já são complacentes com a regulação \cite{ITARPCBCompliance}.

O custo de utilizar técnicas de tolerância é sensível ao contexto da aplicação e ao nível de tolerância desejado, e no caso dos sistemas COTS, técnicas robustas de resiliência em hardware nem sempre estão disponíveis, sendo necessário delegar esta funcionalidade para a aplicação. É necessário escolher a técnica de tolerância mais adequada de uma multidão de técnicas. Para que a escolha seja informada, é essencial que as trocas em termos de performance e grau de confiabilidade adquiridos sejam compreendidos para minimizar o custo de manter uma qualidade de serviço desejada.

\subsection{Solução Proposta}

Implementar e comparar técnicas de escalonamento com detecção de erros, com o objetivo de esclarecer o impacto de performance em relação ao ganho de dependabilidade do sistema, particularmente no contexto de sistemas com restrição \textit{Hard Real Time}, pois se uma técnica é capaz de satisfazer o critério de tempo real mais rígido, também poderá ser usada em contextos com critério temporal mais relaxado.

\section{Objetivos}

Esta seção formaliza os objetivos do trabalho, conforme descrito a seguir.

\bigskip
\subsection{Objetivo Geral}

% TODO: Objetivo geral
Analisar o impacto de técnicas de tolerância à falhas no escalonador de um sistema operacional de tempo real.

\bigskip
\subsection{Objetivos Específicos}

\begin{enumerate}
    \item TODO: Objetivos
\end{enumerate}

\section{Metodologia}

O objetivo do trabalho é descritivo e exploratório, as métricas coletadas são de caráter quantitativo e conclusões e observações derivadas do trabalho serão realizadas de maneira indutiva baseadas nas métricas de performance coletadas e comparadas.

Foi realizado uma pesquisa bibliográfica para a fundamentação e escolha das técnicas e dos materiais do trabalho, sendo esta primariamente focada em autores com obras associadas ao tema de tolerância assim como temas adjacentes relevantes como sistemas operacionais e interface hardware-software.

Após isso será realizado uma implementação e testes das técnicas escolhidas para validação, e uma campanha de injeção de falhas será realizada em um microcontrolador para a coleta final das métricas, este assunto será aprofundado no \autoref{cap:proj}.

\section{Estrutura do Trabalho}

No \autoref{cap:fund} serão explorados os tópicos centrais que constituem a premissa do trabalho, primariamente conceitos de detecção de falhas e escalonamento nos sistemas operacionais. Sistemas operacionais são um tópico particularmente vasto, portanto sua abordagem será focada apenas nos aspectos mais essenciais para o trabalho. No \autoref{cap:proj} é definido o projeto e o plano de verificação para a implementação e análise das técnicas descritas. No \autoref{cap:consid} é realizado uma recapitulação do que foi abordado no trabalho e o que se espera ao final de sua conclusão.


