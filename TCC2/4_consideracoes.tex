\chapter{Considerações Finais}
\label{cap:consid}

Durante a pesquisa bibliográfica, foi possível observar uma variedade de técnicas de tolerância à falhas em software. Das técnicas escolhidas, optou-se por focar em técnicas mais simples para a detecção (CRC e asserts) e técnicas de execução que possibilitem a criação de condições de transparência.

O projeto de pesquisa visa avaliar a aplicação destas técnicas em um contexto de um sistema operacional de tempo real, utilizando de injeção de falhas lógicas em hardware para a construção de uma análise do impacto de diferentes combinações de técnicas. Algumas das hipóteses levantadas foram:

\begin{enumerate}
    \item Asserts terão um impacto pequeno na performance.
    \item Replicação de tarefas introduzirá maior variância no tempo de execução.
    \item Reexecução terá uma variância menor do que a replicação, porém com maior custo de tempo.
    \item Sinais de heartbeat terão um impacto menor na detecção de erros em relação à outras técnicas.
\end{enumerate}

