\chapter{Conclusões}
\label{cap:conclusoes}

Sistemas embarcados são sistemas computacionais com propósito específico e são tipicamente parte de um todo. Dado estas características, é importante que o custo dos recursos necessários para a execução correta de um projeto sejam considerados para a viabilidade do produto final com o máximo de funcionalidade possível com o mínimo de custo de produção e custo energético. O uso de sistemas operacionais de tempo real em sistemas embarcados é popular por fornecer garantias em relação à prioridade de tarefas e permitir um modelo de aplicação assíncrono orientado à eventos.

A dependabilidade é uma propriedade geral do sistema que combina diversos critérios (RAMS) e o aumento da dependabilidade em sistemas de natureza crítica é importante para a segurança dos usuários, outros sistemas também podem se beneficiar ao ofertar uma melhor qualidade de serviço mesmo na presença de adversidades.

Com o fim de melhorar a propriedade de dependabilidade de um sistema operacional de tempo real, foram exploradas técnicas de execução, monitoramento de prazo e de checagem de integridade em cenários de faltas transientes. 

Para analisar o impacto da aplicação das técnicas, foi projetado uma aplicação de teste e uma capanha de injeção de faltas. As técnicas foram implementadas na linguagem C++20 e executadas no microcontrolador STM32F411CEU6 com o sistema operacional FreeRTOS e tiveram suas métricas de performance coletadas.

As técnicas demonstraram um efeito positivo da dependabilidade com diferentes custos de memória e tempo de execução. As técnicas que alteram o fluxo do programa foram capazes de mascarar certas faltas em tempo real, já o uso de CRC foi capaz de permitir recomputar dados após um upset ter alterado um resultado. O monitoramento de prazos foi importante para detectar violações do critério temporal assim como auxiliou durante o processo de desenvolvimento para o diagnóstico de erros lógicos.

Os requisitos funcionais do trabalho puderam ser cumpridos, como descrito no \autoref{tab:verif}. Os requisitos não-funcionais foram majoritariamente cumpridos, apenas com uma ressalva em relação à possibilidade de licença permissiva. O objetivo geral do trabalho também foi atingido ao fornecer uma análise do impacto de técnicas de tolerância à faltas em software.

Para elaboração de trabalhos futuros, é interessante explorar o uso de múltiplos núcleos, assim como expandir a detecção de erros para incluir mecanismos de análise de fluxo de controle a nível de compilador. A campanha de injeção de faltas também poderia ser melhor automatizada com uma ferramenta de instrumentalização como o PyOCD.
