\begin{Info}
% Universidade
{UNIVERSIDADE DO VALE DO ITAJAÍ}
% Escola
{ESCOLA POLITÉCNICA}
% Curso
{CURSO DE CIÊNCIA DA COMPUTAÇÃO}
% Titulo
{TÉCNICAS DE TOLERÂNCIA A FALHAS EM SISTEMAS OPERACIONAIS DE TEMPO REAL}
% Autor
{Marcos Augusto Fehlauer Pereira}
% Cidade e Data
{Itajaí (SC), Dezembro de 2025}
% Nome da Área de concentração
{Sistemas Operacionais}
% Orientador(a)
{Felipe Viel, MSc.}
% Coorientador(a) <Nome do Coorientador(a)>, <Titulação> %%%%%%%% Se não tiver coorientador deixe vazio

\end{Info}

%\begin{Dedicatoria}
% Dedicatória
% \end{Dedicatoria}

% \begin{Agradecimentos}
% Agradeço a todos.
% \end{Agradecimentos}

% \begin{Epigrafe}
% Epígrafe
% \end{Epigrafe}

\begin{Resumo}
Sistemas embarcados são sistemas especializados tipicamente encontrados como um componente lógico de um dispositivo maior. Estes sistemas utilizam com frequência um tipo especializado de sistema operacional: sistemas operacionais de tempo real, que permitem que múltiplas tarefas executem de forma concorrente. Em diversas situações é necessário que esses dispositivos operem em condições adversas, como radiação ionizante e interferência eletromagnética, que alteram seu comportamento esperado e degradam sua qualidade de serviço. Para operar dentro de tais condições, técnicas de tolerância à falhas são aplicadas, visando permitir a operação razoável do sistema mesmo na presença de falhas. Para viabilizar tolerância à falhas é possível utilizar de diversos mecanismos, dentre eles aqueles que operam em conjunto com o escalonador do sistema operacional de tempo real. Este trabalho visa explorar e aplicar técnicas de tolerância e detecção de falhas (Redundância Modular, Reexecução, Heartbeat Signal, CRCs e Asserts) voltadas ao escalonador do sistema operacional, com o objetivo de fornecer uma análise dos custos e benefícios associados a cada combinação de técnicas. Os resultados com injeções de falhas focadas na memória demonstraram que as técnicas implementadas proporcionaram aumento de dependabilidade, sendo que a combinação de reexecução com CRC apresentou melhor equilíbrio entre custo computacional e detecção de falhas.

\textbf{Palavras-Chave}: Sistemas Embarcados. Sistemas Operacionais. Tolerância a Falhas. FreeRTOS. Escalonador.

\end{Resumo}

\begin{Abstract}

\textit{Embedded systems are specialized systems that are typically found as a logical component of a greater device. These systems are frequently equipped with a special kind of operating system: real-time operating systems, which allow for multiple tasks to be executed concurrently. In many situations, it is required that such devices operate in adverse conditions, such as ionizing radiation and electromagnetic interference, which alter their expected behavior and cause a degradation in their quality of service. Thus, to be able to operate within these contexts, fault tolerance techniques are applied, with the goal of allowing reasonable system operation even in the presence of faults. Many mechanisms may be used to achieve fault tolerance, among them those that operate in conjunction with the real-time operating system's scheduler. This work explores and applies fault tolerance and detection techniques (Modular Redundancy, Re-execution, Heartbeat Signal, CRCs and Asserts) that have an interface focused on the operating system's scheduler capabilities, with the main objective of providing an analysis of the tradeoffs attached to each combination of techniques. The obtained results after performing a memory focused fault injection campaign have provided increased dependability, with the combination of re-execution and CRC presenting the best balance between computational cost and fault detection.}

\textit{\textbf{Keywords}: Embedded Systems. Operating Systems. Fault Tolerance. FreeRTOS. Scheduler.}

\end{Abstract}

